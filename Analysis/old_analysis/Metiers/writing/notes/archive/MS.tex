\documentclass[]{article}
\usepackage{lmodern}
\usepackage{amssymb,amsmath}
\usepackage{ifxetex,ifluatex}
\usepackage{fixltx2e} % provides \textsubscript
\ifnum 0\ifxetex 1\fi\ifluatex 1\fi=0 % if pdftex
  \usepackage[T1]{fontenc}
  \usepackage[utf8]{inputenc}
\else % if luatex or xelatex
  \ifxetex
    \usepackage{mathspec}
    \usepackage{xltxtra,xunicode}
  \else
    \usepackage{fontspec}
  \fi
  \defaultfontfeatures{Mapping=tex-text,Scale=MatchLowercase}
  \newcommand{\euro}{€}
\fi
% use upquote if available, for straight quotes in verbatim environments
\IfFileExists{upquote.sty}{\usepackage{upquote}}{}
% use microtype if available
\IfFileExists{microtype.sty}{%
\usepackage{microtype}
\UseMicrotypeSet[protrusion]{basicmath} % disable protrusion for tt fonts
}{}
\usepackage[margin=1in]{geometry}
\usepackage{longtable,booktabs}
\usepackage{graphicx}
\makeatletter
\def\maxwidth{\ifdim\Gin@nat@width>\linewidth\linewidth\else\Gin@nat@width\fi}
\def\maxheight{\ifdim\Gin@nat@height>\textheight\textheight\else\Gin@nat@height\fi}
\makeatother
% Scale images if necessary, so that they will not overflow the page
% margins by default, and it is still possible to overwrite the defaults
% using explicit options in \includegraphics[width, height, ...]{}
\setkeys{Gin}{width=\maxwidth,height=\maxheight,keepaspectratio}
\ifxetex
  \usepackage[setpagesize=false, % page size defined by xetex
              unicode=false, % unicode breaks when used with xetex
              xetex]{hyperref}
\else
  \usepackage[unicode=true]{hyperref}
\fi
\hypersetup{breaklinks=true,
            bookmarks=true,
            pdfauthor={Emma Fuller, Jameal Samhouri, James Watson},
            pdftitle={Markets, ecology and risk: linking human well-being to ecosystems in a commercial fishery system},
            colorlinks=true,
            citecolor=blue,
            urlcolor=blue,
            linkcolor=magenta,
            pdfborder={0 0 0}}
\urlstyle{same}  % don't use monospace font for urls
\setlength{\parindent}{0pt}
\setlength{\parskip}{6pt plus 2pt minus 1pt}
\setlength{\emergencystretch}{3em}  % prevent overfull lines
\setcounter{secnumdepth}{5}

%%% Use protect on footnotes to avoid problems with footnotes in titles
\let\rmarkdownfootnote\footnote%
\def\footnote{\protect\rmarkdownfootnote}

%%% Change title format to be more compact
\usepackage{titling}

% Create subtitle command for use in maketitle
\newcommand{\subtitle}[1]{
  \posttitle{
    \begin{center}\large#1\end{center}
    }
}

\setlength{\droptitle}{-2em}
  \title{Markets, ecology and risk: linking human well-being to ecosystems in a
commercial fishery system}
  \pretitle{\vspace{\droptitle}\centering\huge}
  \posttitle{\par}
  \author{Emma Fuller, Jameal Samhouri, James Watson}
  \preauthor{\centering\large\emph}
  \postauthor{\par}
  \predate{\centering\large\emph}
  \postdate{\par}
  \date{April 14, 2015}



\begin{document}

\maketitle


\section{Abstract}\label{abstract}

Here we link ecological, economic and human wellbeing measures using a
commercial fishery system as an example. We demonstrate how such
analyses can help clarify policy options and focus on characteristics of
the system, ecological or economic that can improve human wellbeing
while keeping ecological integrety in mind.

Variability in returns is a chief concern of anyone making a livelihood
from a natural system, be it farmers or fishermen. As predicted by
porfolio theory, previous work has shown that diversification across
commercial fisheries is negatively related to financial risk to which a
vessel is exposed. Given the clear benefit of diversification across
fisheries, we wonder why all vessels aren't highly diversified.
Differences in diversity among vessels might be due to limited entry
management, but it also might be due to ecological or economic variables
that restrict access to certain species and/or markets. Here we develop
an unsupervised way to sort trips into fisheries based on catch
composition and gear and use fishing port level ecological and market
variables to show that diversity of a vessel is related to both
variables reflecting market accessibility and fish habitat. We discuss
implications of these findings for management.

\section{Introduction}\label{introduction}

As emphasis shifts to a ``people and nature'' framing of conservation,
understanding the linked dynamics between ecosystems and the people who
depend on the system becomes more important for both prediction and
transparent tradeoffs between human well-being and ecological integrity.

Previous work has identified that a fishing vessel's diversification
across fisheries is an indicator of how much financial risk to which a
vessel is exposed. The more diversified a vessel is, the lower income
variability they have, on average. Work in Alaska has suggested that
local geography might be a contributing factor (Sethi, Reimer, and Knapp
2014) or limited entry management (Kasperski and Holland 2013), but to
my knowledge no work has empirically examined what predicts a vessel's
diversification across fisheries. Fisheries diversity could be a product
of a number of social and/or ecological drivers. Limited entry
management is plausible, but so are ecological drivers (are species
present to be fished?) and/or economic constraints in the form of market
availability (is there a place to sell the fish?). This is an important
question because if geography is a general driver of low diversity and
thus high revenue volatility, then there is little policy can do.
However if limited entry management of fisheries is the major driver,
then managers and policy-makers could have a big role in developing
policy improve human wellbeing.

An important detail is that diversity is important if we assume the
fisheries yields are highly variable and asynchronous within and across
years. Within year variability and asynchrony is often due to the
ecology of the species (i.e.~albacore tuna and pacific whiting arrive on
the US west coast only in summer months) or management as is the case
with seasonal fishery openings (i.e.~Dungeness crab). Across year
variability can be due to both management restrictions and recruitment
failures (Chinook salmon has seen catches cut dramatically in some areas
of California\footnote{From interviews, need to check}). Regardless,
there may be species with low-variability life-histories and management.
Targeting these species may negate the need for diversification, and
allow vessels to specialize. Along these lines, there may be pairs or
triplets of fisheries with complementary variation that still reduces
revenue variability at a yearly resolution. Thus before we devote
attention to the understanding of fisheries diversification, we seek to
determine whether diversification's effect on income volatility depends
on the fisheries in which a vessel participates.

We find that many strategies succeed at lowering variability, but
diversity still has a significant negative effect on income variability,
regardless of strategy used. In other words, in each strategy, the
vessels which are most diverse have the lowest variability income. Given
this result, we wonder why don't all vessels diversify given the
reductions in income variability. We hypothesize that the ability to
diversify may depend on the biogeographic locations of catch and market
availability, along with ability to participate in open access
fisheries.

\section{Methods}\label{methods}

\subsection{Data}\label{data}

We use landing receipts from all commercial fisheries on the US west
coast from 2009-2013 provided to us by PacFin. Environmental data layers
come from the essential fish habitat (EFH) review. Data on population of
coastal cities comes from the 2010 Census. Landing receipts include the
following information for each trip: species caught, pounds, price per
pound, date, port of landing, and vessel identifiers. We calculate
revenues by multiplying the price per pound by the number of pounds
reported and adjust revenue to 2009 levels to avoid inflation effects.
Trips were classified as open access or limited entry using the Pacific
Council's classification of groundfish trips and expert opinion (see
Appendix).\footnote{Not complete yet, waiting for the expert opinion.
  Will eventually have a table with fishery and management
  classification and reference (Pacific Council designation, expert
  opinion - and if so, whom)}

\subsection{Metier (fisheries)
definition}\label{metier-fisheries-definition}

Previous work

\begin{itemize}
\itemsep1pt\parskip0pt\parsep0pt
\item
  ({\textbf{???}}) lots of references therein. Need to look about
  previous work.
\end{itemize}

Presently there exists no universal way to define a ``fishery''. In US
West coast commercial landings, the Pacific Management Council has
developed a set of sector based definitions for groundfish landings, but
no equivalent exists for non-groundfish fisheries. In order to treat the
dataset uniformly, we apply a metier analysis to this landing data.

Previous metier analyses occur largely in Europe, although metier-like
analyes have been performed in the Northeast US, classifying fishing
data to define ``operational fisheries'' of New England (Lucey and
Fogarty 2013). While promising as a way to classify fisheries for use in
ecosystem-based management, these methods introduced spatial and
temporal structure prior to defining fisheries. In our analysis such
structure emerges from the data, and we are able to recover the commonly
recognized major fisheries and their seasonality, along with more
spatially and temporally restricted fisheries. These methods have the
additional benefit of only requiring the catch composition of trips,
making it possible to integrate data from both state and federal
management databases which lack permitting data.

A metier is defined as a gear-species target combination (Deporte et al.
2012). We first define species targets and then assign these targets to
gear to make the final metier designation. To find species target data
we classify target species assemblages by first subsetting to all 2010
trips and searching for characteristic catch compositions. To find these
assemblages we first split trips by gear type (using PacFin grgroups
designation) and calcluate a pairwise dissimiliarity index for each trip
within a gear/year subset using the Bray-Curtis dissimilarity index.
This metric has the advantage of avoiding the double-0 problem common in
species count data. The Bray-Curtis dissimilarity index is defined as

\[BC_{ij} = \frac{2C_{ij}}{S_i + S_j}\]

where \(C_{ij}\) is the biomass of the lesser value for only those two
species in common between both sites. Si and Sj are the total number of
individuals counted in both trips. This index ranges between zero and
one, with zero meaning the sites have the same composition and one
meaning they share no species.

We transform the disimilarity index to be a measure of similarity

\[\text{Similarity} = 1-\mid BC \mid\]

and build an undirected, weighted network in which nodes are trips, and
edge widths are the similarity in species composition between trips.
This allows a vessel to be represented in multiple nodes if it makes
trips that vary substantially in catch composition. With this network we
use the \emph{infoMap} algorithm to find communities (clusters or
subgraphs) within the network (using the implementation in the
\texttt{R} package \texttt{igraph})(Martin Rosvall and Bergstrom 2008; M
Rosvall, Axelsson, and Bergstrom 2009).

\emph{infoMap} is an information theoretic approach, which uses the
probability flow of a random walker on a network as a proxy for the
information flows in a real system. The objective of infoMap is to
compress the description of the probability flow, and in doing so
partitions the network into modules. \emph{infoMap} works by computing
the fraction of time a node is visted by a random walker using a
deterministically greedy search algorithm. Merges between modules that
give the largest decrease in description length are made until further
merging leads to increases of description length. Results are refined
with a simulated annealing approach, starting at several different
temperatures, with the run selected as the one that gives the shortest
description of the network.

We found that other commonly used clustering algorithms (i.e.~k-means,
hierarchical clustering) did poorly with this data. Many clustering
algorithms do best when clusters are spherical in n-dimensional space,
and/or require the number of clusters decided \emph{a priori}. In this
data we have fisheries participation may vary by several orders of
magnitude (100s of trips to 100,000s of trips), and we wanted to avoid
having to decide subjectively on the number of clusters.

After dropping any modules that have fewer than five trips, we use a
k-nearest-neighbor (knn) classifier to assign all other trips of each
gear subset to those possible metiers. The nearest neighbor to each trip
was found using the Bray-Curtis dissimilarity index (transformed into a
similarity) and all analyses were performed using \texttt{R}.

\subsection{Revenue diversity}\label{revenue-diversity}

We measure revenue diversity (or portfolio diversification) with the
Simpson's diversity index (Kasperski and Holland 2013; Sethi, Reimer,
and Knapp 2014). Simpson's diversity index is calculated per vessel;
i.e.~have an index of \(j = 1,\dots,N\) where \(N\) is the number of
vessels. Thus \(S_j\) is defined as \[ S_j = 1 - \sum^k_{i=1}p_i^2\] for
\(k\) fisheries with \(p_i\) as the proportion of total gross revenue
from fishery \(i\). Here values of \(0\) indicate no diversity (a single
fishery), where values close to \(1\) indicate high levels of diversity.
We calculate the diversity indices each year a vessel is active and
averaged across years. Diversity indices are calculated using the
\texttt{vegan()} package in \texttt{R}. Vessels had to be active for at
least two years in order to be included in this analysis.

\subsection{Revenue variability}\label{revenue-variability}

Revenue variability is calculated as the coefficient of variation in
annual revenues. For a single vessel \(j\), the coefficient of variation
(\(CV\)) is calculated as \[CV_j = \frac{\text{sd}(x)}{\text{mean}(x)}\]
where \(x\) is a vector of the annual revenues for vessel \(j\).

\subsection{Strategy definition}\label{strategy-definition}

I define ``vessel strategy'' to be the composition of fisheries from
which the vessel gets revenue. 100\% of revenue from dungeness crab pots
is an example of one strategy, 70\% from tuna trolling and 30\% from
crab is another. This definition has a yearly resolution, thus a vessel
could participate in a different strategy each year. For now, if a
vessel participated in more than one strategy across five years, I
classify it as \texttt{multi}.

The number of vessels using each strategy varies. The Dungeness-crabbing
strategy is the largest with hundreds of vessels. At the other end of
the spectrum a number of strategies are quite rare, with 1-4 vessels
engaged. These strategies are often made up of marginal fisheries
characterized by catches of species like ``unspecified octopus'' or
``unspecified mollusks''. Further, confidentially agreeements require us
to present results which aggregate data from no fewer than 3 vessels.
Thus we do not consider any strategy that has fewer than 3 vessels.
Finally, in order to examine inter-annual revenue variablity I need, at
minimum, two years of landings data. To be conservative, I only examine
vessels for which I have a full five years worth of data.

In this population of fishermen there is a large mass point at very low
revenue diversity. Because we are interested in the how independent
economic and environmental covariates affect the variation in diversity,
we subset to vessels which participate in a yearly median number of
fisheries greater than 1.

After filtering for vessels with a minimum of \$10,000 median revenues
across 5 years of landings and removing vessels which participate in
rare strategies I'm left with 850 vessels and 186105 trips and I find 96
metiers and 19 strategies. This population thus represent approximately
56\% of the commercial vessels and brings in about 68\% of the total
revenue in the 5 year period.

\subsection{Port level variables: markets and
ecology}\label{port-level-variables-markets-and-ecology}

Coming soon.

Landscape covariates were obtained by first plotting 100 km buffers
around each port latitude and longitude. These port polygons were then
overlaid with the GSCGH coastline polygon to remove any portion of the
port polygon that was ``on land''. These port polygons were then
overlaid on landscape covariates of interest.

\subsection{Statistical models}\label{statistical-models}

I square-root transform the revenue variability to achieve a normal
distribution and use a linear regression to examine to quantify the
relationship between revenue variability and mean annual revenue
diversity. For models in which revenue diversity is the dependent
variable I use generalized linear models (glms) with a biomodal logit
link function.{[}\^{}4{]} This is because the revenue diversity is
bounded between zero and one.

To interepret the main effects in the presence of interactions, we
mean-centered continous covariates.

\section{Results}\label{results}

\subsection{Strategy and diversity}\label{strategy-and-diversity}

We replicate earlier findings that diversity of fisheries reduces income
variation (Figure 1). We find that there is support for at least a few
of the strategies to have a significantly different intercept than then
strategy 1 (which is used as a baseline). Almost all strategy
coefficients are significantly negative, which means they are associated
with a decrease in mean variability relative to strategy 1 (except for
\texttt{multi}). Because diversity remains significant, this means that
regardless of what fishing strategy a vessel chooses, the more diverse
within that strategy the vessel is, the lower the inter-annual variation
in revenue (Tables 1 and 2). Including strategy identity significantly
improves the model (AIC of original regression: \(-699.2021\), AIC of
strategy regression: \(-797.4598\)). We also investigate whether
diversity operates in a similar way across for all strategies by testing
whether interactions would improve this model. While there is support
for interactions between diversity and strategy being important we find
that it does not lower the AIC score (AIC of interaction model:
\(-291.9023\)).

For strategies that have a negative relationship between diversity and
revenue variability we are left with the question of what predicts
diversity of a vessel. Why wouldn't all vessels increase fisheries
diversity slide? We switch our focus to evaluating the predictors of
vessel diversity.

\subsection{Influence of markets, ecology and management on vessel
diversity}\label{influence-of-markets-ecology-and-management-on-vessel-diversity}

We first test whether economic covariates (distance to large cities or
number of first receivers) predicts vessel diversity. We find that the
number of first receivers is significantly postively related to the
fisheries diversity of vessels, i.e.~the more first receivers a port has
at which a vessel lands, the more diverse those vessels tend to be
(figure 3). We do not find any support for the distance from large
cities as related to vessel diversity (figure 4).

Macroecological characteristics of where a vessel fishes (proxied by the
location of a port where the vessel lands) might also affect fisheries
diversity. We test whether diversity of habitat types, absolute amount
of rocky habitat or distance to deep water predicts fisheries diversity.
I find that the diversity of habitat types (rocky, mixed, soft) is
strongly and positively related to mean annual fisheries diversity
(figure 5). The percentage of that habitat which is classified by the
EFH as ``hard substrate'' is not postively and signficiantly related to
fisheries diversity (figure 6). We also find that while distance to
upper shelf is not significantly related to diversity (figures 7),
distance to the boundary of the lower shelf is negatively and
significantly related to fisheries diversity, i.e.~the further away from
deep water a vessel tends to land, the less diverse the vessel is
(figure 8).

Finally we relate the presence of open access (OA) and limited entry
(LE) fisheries to vessel diversity. Specifically, we hypothesize that
the proportion of trips which are landed as open access is a proxy for
measuring unobservable port characteristics, namely that the larger the
percentage of total trips which OA makes up, the more viable those
fisheries are. We find that as the percentage of OA trips increases
vessels are \_\_\_\_ more diverse.\footnote{Haven't finished classifying
  these fisheries yet}

\section{Discussion}\label{discussion}

Coming soon

\section{further questions and notes}\label{further-questions-and-notes}

\begin{itemize}
\itemsep1pt\parskip0pt\parsep0pt
\item
  Which strategies/fisheries are the ``best off'' (highest revenues,
  lowest variability) using? The worst?
\item
  How to relate to management: open access probably matters most for
  vessels that can't afford to diversify into capital intensive
  fisheries (i.e.~crab, tuna or salmon). Would expect poorer boats in
  open access to benefit more than richer boats
\item
  Could look through strategies: i.e. \texttt{TWL\_1} strategy as
  compared to \texttt{TLS\_1} \texttt{HKL\_1}, \texttt{HKL\_3} type
  strategies. Would expect \texttt{HKL\_3} to need acess to markets and
  habitat, not as important for \texttt{TWL\_1}.
\end{itemize}

\section{Appendix}\label{appendix}

Here I include figures and model outputs.

\begin{figure}[htbp]
\centering
\includegraphics{MS_files/figure-latex/fig1-1.pdf}
\caption{Fisheries diversity is measured by the average of the annual
Simpsons index of diversity calculated using a vessel's yearly revenues
from each fishery. Income variability is measured by the coefficient of
variation across annual total revenues. The blue line is the linear
regression of
\texttt{income variability \textasciitilde{} fisheries diversity} which
has a significant slope (p = 0)}
\end{figure}

\begin{longtable}[c]{@{}lllll@{}}
\caption{Fitting generalized (gaussian/identity) linear model:
sqrt(cv\_adj\_revenue) \textasciitilde{} mean\_simpson +
single\_cluster}\tabularnewline
\toprule
\begin{minipage}[b]{0.31\columnwidth}\raggedright\strut
~
\strut\end{minipage} &
\begin{minipage}[b]{0.13\columnwidth}\raggedright\strut
Estimate
\strut\end{minipage} &
\begin{minipage}[b]{0.16\columnwidth}\raggedright\strut
Std. Error
\strut\end{minipage} &
\begin{minipage}[b]{0.12\columnwidth}\raggedright\strut
t value
\strut\end{minipage} &
\begin{minipage}[b]{0.12\columnwidth}\raggedright\strut
Pr(\textgreater{}\textbar{}t\textbar{})
\strut\end{minipage}\tabularnewline
\midrule
\endfirsthead
\toprule
\begin{minipage}[b]{0.31\columnwidth}\raggedright\strut
~
\strut\end{minipage} &
\begin{minipage}[b]{0.13\columnwidth}\raggedright\strut
Estimate
\strut\end{minipage} &
\begin{minipage}[b]{0.16\columnwidth}\raggedright\strut
Std. Error
\strut\end{minipage} &
\begin{minipage}[b]{0.12\columnwidth}\raggedright\strut
t value
\strut\end{minipage} &
\begin{minipage}[b]{0.12\columnwidth}\raggedright\strut
Pr(\textgreater{}\textbar{}t\textbar{})
\strut\end{minipage}\tabularnewline
\midrule
\endhead
\begin{minipage}[t]{0.31\columnwidth}\raggedright\strut
\textbf{mean\_simpson}
\strut\end{minipage} &
\begin{minipage}[t]{0.13\columnwidth}\raggedright\strut
-0.256
\strut\end{minipage} &
\begin{minipage}[t]{0.16\columnwidth}\raggedright\strut
0.036
\strut\end{minipage} &
\begin{minipage}[t]{0.12\columnwidth}\raggedright\strut
-7.131
\strut\end{minipage} &
\begin{minipage}[t]{0.12\columnwidth}\raggedright\strut
0
\strut\end{minipage}\tabularnewline
\begin{minipage}[t]{0.31\columnwidth}\raggedright\strut
\textbf{single\_cluster11}
\strut\end{minipage} &
\begin{minipage}[t]{0.13\columnwidth}\raggedright\strut
-0.225
\strut\end{minipage} &
\begin{minipage}[t]{0.16\columnwidth}\raggedright\strut
0.045
\strut\end{minipage} &
\begin{minipage}[t]{0.12\columnwidth}\raggedright\strut
-5.04
\strut\end{minipage} &
\begin{minipage}[t]{0.12\columnwidth}\raggedright\strut
0
\strut\end{minipage}\tabularnewline
\begin{minipage}[t]{0.31\columnwidth}\raggedright\strut
\textbf{single\_cluster12}
\strut\end{minipage} &
\begin{minipage}[t]{0.13\columnwidth}\raggedright\strut
-0.044
\strut\end{minipage} &
\begin{minipage}[t]{0.16\columnwidth}\raggedright\strut
0.067
\strut\end{minipage} &
\begin{minipage}[t]{0.12\columnwidth}\raggedright\strut
-0.651
\strut\end{minipage} &
\begin{minipage}[t]{0.12\columnwidth}\raggedright\strut
0.516
\strut\end{minipage}\tabularnewline
\begin{minipage}[t]{0.31\columnwidth}\raggedright\strut
\textbf{single\_cluster13}
\strut\end{minipage} &
\begin{minipage}[t]{0.13\columnwidth}\raggedright\strut
-0.001
\strut\end{minipage} &
\begin{minipage}[t]{0.16\columnwidth}\raggedright\strut
0.049
\strut\end{minipage} &
\begin{minipage}[t]{0.12\columnwidth}\raggedright\strut
-0.029
\strut\end{minipage} &
\begin{minipage}[t]{0.12\columnwidth}\raggedright\strut
0.977
\strut\end{minipage}\tabularnewline
\begin{minipage}[t]{0.31\columnwidth}\raggedright\strut
\textbf{single\_cluster15}
\strut\end{minipage} &
\begin{minipage}[t]{0.13\columnwidth}\raggedright\strut
0.023
\strut\end{minipage} &
\begin{minipage}[t]{0.16\columnwidth}\raggedright\strut
0.075
\strut\end{minipage} &
\begin{minipage}[t]{0.12\columnwidth}\raggedright\strut
0.303
\strut\end{minipage} &
\begin{minipage}[t]{0.12\columnwidth}\raggedright\strut
0.762
\strut\end{minipage}\tabularnewline
\begin{minipage}[t]{0.31\columnwidth}\raggedright\strut
\textbf{single\_cluster16}
\strut\end{minipage} &
\begin{minipage}[t]{0.13\columnwidth}\raggedright\strut
0.001
\strut\end{minipage} &
\begin{minipage}[t]{0.16\columnwidth}\raggedright\strut
0.062
\strut\end{minipage} &
\begin{minipage}[t]{0.12\columnwidth}\raggedright\strut
0.017
\strut\end{minipage} &
\begin{minipage}[t]{0.12\columnwidth}\raggedright\strut
0.986
\strut\end{minipage}\tabularnewline
\begin{minipage}[t]{0.31\columnwidth}\raggedright\strut
\textbf{single\_cluster2}
\strut\end{minipage} &
\begin{minipage}[t]{0.13\columnwidth}\raggedright\strut
-0.078
\strut\end{minipage} &
\begin{minipage}[t]{0.16\columnwidth}\raggedright\strut
0.034
\strut\end{minipage} &
\begin{minipage}[t]{0.12\columnwidth}\raggedright\strut
-2.315
\strut\end{minipage} &
\begin{minipage}[t]{0.12\columnwidth}\raggedright\strut
0.021
\strut\end{minipage}\tabularnewline
\begin{minipage}[t]{0.31\columnwidth}\raggedright\strut
\textbf{single\_cluster3}
\strut\end{minipage} &
\begin{minipage}[t]{0.13\columnwidth}\raggedright\strut
0.047
\strut\end{minipage} &
\begin{minipage}[t]{0.16\columnwidth}\raggedright\strut
0.04
\strut\end{minipage} &
\begin{minipage}[t]{0.12\columnwidth}\raggedright\strut
1.19
\strut\end{minipage} &
\begin{minipage}[t]{0.12\columnwidth}\raggedright\strut
0.234
\strut\end{minipage}\tabularnewline
\begin{minipage}[t]{0.31\columnwidth}\raggedright\strut
\textbf{single\_cluster4}
\strut\end{minipage} &
\begin{minipage}[t]{0.13\columnwidth}\raggedright\strut
-0.119
\strut\end{minipage} &
\begin{minipage}[t]{0.16\columnwidth}\raggedright\strut
0.051
\strut\end{minipage} &
\begin{minipage}[t]{0.12\columnwidth}\raggedright\strut
-2.342
\strut\end{minipage} &
\begin{minipage}[t]{0.12\columnwidth}\raggedright\strut
0.019
\strut\end{minipage}\tabularnewline
\begin{minipage}[t]{0.31\columnwidth}\raggedright\strut
\textbf{single\_cluster5}
\strut\end{minipage} &
\begin{minipage}[t]{0.13\columnwidth}\raggedright\strut
-0.103
\strut\end{minipage} &
\begin{minipage}[t]{0.16\columnwidth}\raggedright\strut
0.027
\strut\end{minipage} &
\begin{minipage}[t]{0.12\columnwidth}\raggedright\strut
-3.788
\strut\end{minipage} &
\begin{minipage}[t]{0.12\columnwidth}\raggedright\strut
0
\strut\end{minipage}\tabularnewline
\begin{minipage}[t]{0.31\columnwidth}\raggedright\strut
\textbf{single\_cluster6}
\strut\end{minipage} &
\begin{minipage}[t]{0.13\columnwidth}\raggedright\strut
-0.176
\strut\end{minipage} &
\begin{minipage}[t]{0.16\columnwidth}\raggedright\strut
0.026
\strut\end{minipage} &
\begin{minipage}[t]{0.12\columnwidth}\raggedright\strut
-6.669
\strut\end{minipage} &
\begin{minipage}[t]{0.12\columnwidth}\raggedright\strut
0
\strut\end{minipage}\tabularnewline
\begin{minipage}[t]{0.31\columnwidth}\raggedright\strut
\textbf{single\_cluster7}
\strut\end{minipage} &
\begin{minipage}[t]{0.13\columnwidth}\raggedright\strut
-0.055
\strut\end{minipage} &
\begin{minipage}[t]{0.16\columnwidth}\raggedright\strut
0.042
\strut\end{minipage} &
\begin{minipage}[t]{0.12\columnwidth}\raggedright\strut
-1.29
\strut\end{minipage} &
\begin{minipage}[t]{0.12\columnwidth}\raggedright\strut
0.197
\strut\end{minipage}\tabularnewline
\begin{minipage}[t]{0.31\columnwidth}\raggedright\strut
\textbf{single\_cluster8}
\strut\end{minipage} &
\begin{minipage}[t]{0.13\columnwidth}\raggedright\strut
-0.178
\strut\end{minipage} &
\begin{minipage}[t]{0.16\columnwidth}\raggedright\strut
0.04
\strut\end{minipage} &
\begin{minipage}[t]{0.12\columnwidth}\raggedright\strut
-4.489
\strut\end{minipage} &
\begin{minipage}[t]{0.12\columnwidth}\raggedright\strut
0
\strut\end{minipage}\tabularnewline
\begin{minipage}[t]{0.31\columnwidth}\raggedright\strut
\textbf{single\_cluster9}
\strut\end{minipage} &
\begin{minipage}[t]{0.13\columnwidth}\raggedright\strut
-0.069
\strut\end{minipage} &
\begin{minipage}[t]{0.16\columnwidth}\raggedright\strut
0.029
\strut\end{minipage} &
\begin{minipage}[t]{0.12\columnwidth}\raggedright\strut
-2.369
\strut\end{minipage} &
\begin{minipage}[t]{0.12\columnwidth}\raggedright\strut
0.018
\strut\end{minipage}\tabularnewline
\begin{minipage}[t]{0.31\columnwidth}\raggedright\strut
\textbf{single\_clustermulti}
\strut\end{minipage} &
\begin{minipage}[t]{0.13\columnwidth}\raggedright\strut
0.036
\strut\end{minipage} &
\begin{minipage}[t]{0.16\columnwidth}\raggedright\strut
0.012
\strut\end{minipage} &
\begin{minipage}[t]{0.12\columnwidth}\raggedright\strut
2.976
\strut\end{minipage} &
\begin{minipage}[t]{0.12\columnwidth}\raggedright\strut
0.003
\strut\end{minipage}\tabularnewline
\begin{minipage}[t]{0.31\columnwidth}\raggedright\strut
\textbf{(Intercept)}
\strut\end{minipage} &
\begin{minipage}[t]{0.13\columnwidth}\raggedright\strut
0.718
\strut\end{minipage} &
\begin{minipage}[t]{0.16\columnwidth}\raggedright\strut
0.014
\strut\end{minipage} &
\begin{minipage}[t]{0.12\columnwidth}\raggedright\strut
51.7
\strut\end{minipage} &
\begin{minipage}[t]{0.12\columnwidth}\raggedright\strut
0
\strut\end{minipage}\tabularnewline
\bottomrule
\end{longtable}

\begin{longtable}[c]{@{}llllll@{}}
\caption{Analysis of Variance Model}\tabularnewline
\toprule
\begin{minipage}[b]{0.36\columnwidth}\raggedright\strut
~
\strut\end{minipage} &
\begin{minipage}[b]{0.05\columnwidth}\raggedright\strut
Df
\strut\end{minipage} &
\begin{minipage}[b]{0.10\columnwidth}\raggedright\strut
Sum Sq
\strut\end{minipage} &
\begin{minipage}[b]{0.11\columnwidth}\raggedright\strut
Mean Sq
\strut\end{minipage} &
\begin{minipage}[b]{0.11\columnwidth}\raggedright\strut
F value
\strut\end{minipage} &
\begin{minipage}[b]{0.11\columnwidth}\raggedright\strut
Pr(\textgreater{}F)
\strut\end{minipage}\tabularnewline
\midrule
\endfirsthead
\toprule
\begin{minipage}[b]{0.36\columnwidth}\raggedright\strut
~
\strut\end{minipage} &
\begin{minipage}[b]{0.05\columnwidth}\raggedright\strut
Df
\strut\end{minipage} &
\begin{minipage}[b]{0.10\columnwidth}\raggedright\strut
Sum Sq
\strut\end{minipage} &
\begin{minipage}[b]{0.11\columnwidth}\raggedright\strut
Mean Sq
\strut\end{minipage} &
\begin{minipage}[b]{0.11\columnwidth}\raggedright\strut
F value
\strut\end{minipage} &
\begin{minipage}[b]{0.11\columnwidth}\raggedright\strut
Pr(\textgreater{}F)
\strut\end{minipage}\tabularnewline
\midrule
\endhead
\begin{minipage}[t]{0.36\columnwidth}\raggedright\strut
\textbf{mean\_simpson}
\strut\end{minipage} &
\begin{minipage}[t]{0.05\columnwidth}\raggedright\strut
1
\strut\end{minipage} &
\begin{minipage}[t]{0.10\columnwidth}\raggedright\strut
1.149
\strut\end{minipage} &
\begin{minipage}[t]{0.11\columnwidth}\raggedright\strut
1.149
\strut\end{minipage} &
\begin{minipage}[t]{0.11\columnwidth}\raggedright\strut
29.32
\strut\end{minipage} &
\begin{minipage}[t]{0.11\columnwidth}\raggedright\strut
0
\strut\end{minipage}\tabularnewline
\begin{minipage}[t]{0.36\columnwidth}\raggedright\strut
\textbf{single\_cluster}
\strut\end{minipage} &
\begin{minipage}[t]{0.05\columnwidth}\raggedright\strut
5
\strut\end{minipage} &
\begin{minipage}[t]{0.10\columnwidth}\raggedright\strut
2.972
\strut\end{minipage} &
\begin{minipage}[t]{0.11\columnwidth}\raggedright\strut
0.594
\strut\end{minipage} &
\begin{minipage}[t]{0.11\columnwidth}\raggedright\strut
15.18
\strut\end{minipage} &
\begin{minipage}[t]{0.11\columnwidth}\raggedright\strut
0
\strut\end{minipage}\tabularnewline
\begin{minipage}[t]{0.36\columnwidth}\raggedright\strut
\textbf{mean\_simpson:single\_cluster}
\strut\end{minipage} &
\begin{minipage}[t]{0.05\columnwidth}\raggedright\strut
5
\strut\end{minipage} &
\begin{minipage}[t]{0.10\columnwidth}\raggedright\strut
0.516
\strut\end{minipage} &
\begin{minipage}[t]{0.11\columnwidth}\raggedright\strut
0.103
\strut\end{minipage} &
\begin{minipage}[t]{0.11\columnwidth}\raggedright\strut
2.635
\strut\end{minipage} &
\begin{minipage}[t]{0.11\columnwidth}\raggedright\strut
0.023
\strut\end{minipage}\tabularnewline
\begin{minipage}[t]{0.36\columnwidth}\raggedright\strut
\textbf{Residuals}
\strut\end{minipage} &
\begin{minipage}[t]{0.05\columnwidth}\raggedright\strut
749
\strut\end{minipage} &
\begin{minipage}[t]{0.10\columnwidth}\raggedright\strut
29.34
\strut\end{minipage} &
\begin{minipage}[t]{0.11\columnwidth}\raggedright\strut
0.039
\strut\end{minipage} &
\begin{minipage}[t]{0.11\columnwidth}\raggedright\strut
NA
\strut\end{minipage} &
\begin{minipage}[t]{0.11\columnwidth}\raggedright\strut
NA
\strut\end{minipage}\tabularnewline
\bottomrule
\end{longtable}

\begin{figure}[htbp]
\centering
\includegraphics{MS_files/figure-latex/unnamed-chunk-3-1.pdf}
\caption{Fisheries diversity is measured by the average of the annual
Simpsons index of diversity calculated using a vessel's yearly revenues
from each fishery. Income variability is measured by the coefficient of
variation across annual total revenues. The colored lines are the
predicted values for each strategy for the fixed effects model which
uses strategy identity.}
\end{figure}

\begin{figure}[htbp]
\centering
\includegraphics{MS_files/figure-latex/unnamed-chunk-4-1.pdf}
\caption{Number of first receivers is measured as the mean yearly number
of first receivers at the port level. Each vessel is assigned a score
based on a revenue-weighted average based on landings within a year.
Fishery diversity is defined as mean annual Simpsons index of fisheries
revenue. The teal line on the left is the glm model fit of of
\texttt{fisheries diversity \textasciitilde{} first receivers} (p.value
\(= 0.00407\)) plotted over the raw data. The plot on the right is a
boxplot version of the data plotted on the left.}
\end{figure}

\begin{figure}[htbp]
\centering
\includegraphics{MS_files/figure-latex/unnamed-chunk-5-1.pdf}
\caption{Minimum distance to a big city is measured at the port level as
the minimum distance to a city with a population \(>100000\) according
to the 2010 census. Each vessel is assigned a score based on a
revenue-weighted average based on landings within a year. Fishery
diversity is defined as mean annual Simpsons index of fisheries revenue.
The pink line on the left is the glm model fit of
\texttt{fisheries diversity \textasciitilde{} first receivers}, (p.value
\(= 0.616\)). The plot on the right is a boxplot version of the data
plotted on the left.}
\end{figure}

\begin{figure}[htbp]
\centering
\includegraphics{MS_files/figure-latex/unnamed-chunk-6-1.pdf}
\caption{Habitat diversity is measured as the Simpon's index of area of
each of three possible types of habitat: soft, mixed and hard bottom
substrate. Each vessel is assigned a score based on a revenue-weighted
average based on landings within a year, averaged across years. Fishery
diversity is defined as mean annual Simpsons index of fisheries revenue.
The green line on the left is the glm model fit of
\texttt{fisheries diversity \textasciitilde{} habitat diversity},
(p.value \(= 0.0032\)). The plot on the right is a boxplot version of
the data plotted on the left.}
\end{figure}

\begin{figure}[htbp]
\centering
\includegraphics{MS_files/figure-latex/unnamed-chunk-7-1.pdf}
\caption{Percent rocky habitat is calculated as the percent of substrate
as categorized by EFH that is `hard substrate' within a 100km radius
from the port. Note, different overall percentages of substrate are
categorized by EFH depending on the port geography. Each vessel is
assigned a score based on a revenue-weighted average based on landings
within a year, averaged across years. Fishery diversity is defined as
mean annual Simpsons index of fisheries revenue. The purple line on the
left is the glm model fit of
\texttt{fisheries diversity \textasciitilde{} percent rocky habitat}, p
value \(=0.169\). The plot on the right is a boxplot version of the data
plotted on the left.}
\end{figure}

\begin{figure}[htbp]
\centering
\includegraphics{MS_files/figure-latex/unnamed-chunk-8-1.pdf}
\caption{Minimum distance to between port of landing and upper shelf
break is calculated using EFH's designation of upper and lower shelf
breaks. Each vessel is assigned a score based on a revenue-weighted
average of landings within a year, averaged across years. Fishery
diversity is defined as mean annual Simpsons index of fisheries revenue.
The yellow line on the left is the glm model fit of
\texttt{fisheries diversity \textasciitilde{} distance to upper slope},
p.value \(=0.0897\). The plot on the right is a boxplot version of the
data plotted on the left.}
\end{figure}

\begin{figure}[htbp]
\centering
\includegraphics{MS_files/figure-latex/unnamed-chunk-9-1.pdf}
\caption{Minimum distance to between port of landing and lower shelf
break is calculated using EFH's designation of upper and lower shelf
breaks. Each vessel is assigned a score based on a revenue-weighted
average of landings within a year, averaged across years. Fishery
diversity is defined as mean annual Simpsons index of fisheries revenue.
The red line on the left is the glm model fit of
\texttt{fisheries diversity \textasciitilde{} distance to lower slope},
p.value \(=5.23e-10\). The plot on the right is a boxplot version of the
data plotted on the left.}
\end{figure}

\begin{figure}[htbp]
\centering
\includegraphics{MS_files/figure-latex/unnamed-chunk-10-1.pdf}
\caption{Percent cover of MPAs are calculated using EFH's designation
MPAs that have some restrictions or prohibitions of commercial fishing
of any time. Each vessel is assigned a score based on a revenue-weighted
average of landings within a year, averaged across years. Fishery
diversity is defined as mean annual Simpsons index of fisheries revenue.
The red line on the left is the glm model fit of
\texttt{fisheries diversity \textasciitilde{} distance to lower slope},
p.value \(=\). The plot on the right is a boxplot version of the data
plotted on the left.}
\end{figure}

\section*{References}\label{references}
\addcontentsline{toc}{section}{References}

Deporte, N, C Ulrich, S Mahevas, S Demaneche, and F Bastardie. 2012.
``Regional metier definition: a comparative investigation of statistical
methods using a workflow applied to international otter trawl fisheries
in the North Sea.'' \emph{ICES Journal of Marine Science} 69 (2):
331--42.

Kasperski, S, and D S Holland. 2013. ``Income diversification and risk
for fishermen.'' \emph{Proceedings of the National Academy of Sciences}.

Lucey, S M, and M J Fogarty. 2013. ``Operational fisheries in New
England: Linking current fishing patterns to proposed ecological
production units.'' \emph{Fisheries Research}.

Rosvall, M, D Axelsson, and C T Bergstrom. 2009. ``The map equation.''
\emph{The European Physical Journal Special Topics} 178 (1): 13--23.

Rosvall, Martin, and Carl T Bergstrom. 2008. ``Maps of random walks on
complex networks reveal community structure.'' \emph{Proceedings of the
National Academy of Sciences} 105 (4): 1118--23.

Sethi, S A, M Reimer, and G Knapp. 2014. ``Alaskan fishing community
revenues and the stabilizing role of fishing portfolios.'' \emph{Marine
Policy}.

\end{document}

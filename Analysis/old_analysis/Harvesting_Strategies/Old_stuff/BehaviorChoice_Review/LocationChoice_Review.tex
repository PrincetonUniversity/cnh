\documentclass[11pt, oneside]{article}   	% use "amsart" instead of "article" for AMSLaTeX format
\usepackage{geometry}                		% See geometry.pdf to learn the layout options. There are lots.
\geometry{letterpaper}                   		% ... or a4paper or a5paper or ... 
%\geometry{landscape}                		% Activate for for rotated page geometry
%\usepackage[parfill]{parskip}    		% Activate to begin paragraphs with an empty line rather than an indent
\usepackage{graphicx}				% Use pdf, png, jpg, or eps� with pdflatex; use eps in DVI mode
\usepackage{natbib}								% TeX will automatically convert eps --> pdf in pdflatex		
\usepackage{amssymb}

\title{Location Choice Review}
\author{Emma Fuller}
%\date{}							% Activate to display a given date or no date

\begin{document}
\maketitle

\section{There are many calls for better understanding fishermen behavior.}
"Understanding the behavior of fishermen is essential if we wish to examine the impact of management regimes like ITQs and MPAs on fisheries." \citep{Branchetal2005}.

"We hope that a focus on fisher-men behavior will address the comments made by (Hilborn 1985) about fisheries management: �This sad litany of fish-eries disasters can be ascribed to poor understanding of the dynamics of fishermen, how they fish, and how they invest�. Fisheries scientists still largely concentrate on fish and less on fishermen. Fishing opportunities provide a way of re-dressing this imbalance, thus hopefully ensuring that the right numbers of fishermen are allowed to chase after the right number of fish in the right places." \citep{Branchetal2005}

Common observation that "research users behave in a  manner that is often unintended by the designers of the management system." \cite{Fultonetal2011}
"...the uncertainty generated by unexpected resource user behavior is critical as it has unplanned consequences and leads to unintended management outcomes." \citep{Fultonetal2011}
'
"It is a well-known truism that resource managers attempt to manage people not fish (Jentoft 1997)" \citep{Fultonetal2011}. 

"It is also true that resource users can be manipulated to behave in ways that achieve desired management outcomes. However, the past has shown us that frequently the implementation of policies has had unexpected, and sometimes undesirable outcomes." \citep{Fultonetal2011}

Should focus on fisher behavior because "It is our contention that uncertainty generated by the divergence of the intended and actual outcomes of management actions has become the major source of uncertainty in the adaptive management cycle." \citep{Fultonetal2011}

\subsection{What is the definition of fishing behavior?}
"A large part of fishing behavior is choosing where to fish" \citep{Branchetal2005}. 
"One of the most basic elements of fishing behavior is the choice of when and where to de-ploy fishing gear: a skipper chooses between conducting exploratory fishing and fishing at competing fishing opportunities. We define fishing opportunities to be small areas over which the skipper of a vessel frequently trawls, i.e., groups of trawls that are consistently placed over the same geographic area." \citep{Branchetal2005}.
"As defined above, fishing opportunities are similar to what might be called fishing grounds. However, we con-sider fishing grounds to be larger regions that are fished by many vessels, whereas fishing opportunities represent vessel-specific methods of catching fish within these fishing grounds. Thus, for each fishing opportunity, a skipper will have previous knowledge gained through fishing (or from other skippers) about likely catch rates, species mix, optimal fishing months, appropriate gear, and the probable costs of fishing. Based on their knowl-edge, one skipper might only conduct short trawls in summer in the northern part of a fishing ground to target a skate species, while another vessel fishing in the same fishing ground may always conduct long winter trawls that traverse most of fishing ground and catch a wide mix of species. In this paper, we therefore define fishing opportunities to be vessel specific so that they can be re-lated directly to fishermen behavior. In reality, fishing grounds often include areas that are untrawlable (i.e., too steep or rough), limiting the number of available fishing opportunities." \citep{Branchetal2005}. 
"How do skippers choose where to fish? The answer de-pends on the scale of the fishing areas that are considered and how they are defined. In most fisheries, skippers can tell you the names of their favorite fishing locations, places where they return to year after year because they know catch rates of their target species will be high there. We call these locations fishing opportunities. At each fishing opportunity, the mix of species caught depends on the bottom topography (steep or flat) and the substrate type (e.g., rocky or muddy) of that location. Skippers learn which gear to use and what time of the year to fish at each fishing opportunity; there-fore, fishing opportunities are best defined as vessel specific. Furthermore, since learning is a function of historical search patterns and exploratory behavior, each skipper will have in-formation about a specific set of fishing opportunities, al-though information sharing among skippers may broaden the range of available fishing opportunities.

\section{Fishing behavior has been treated in a variety of ways}
The literature covering harvesting strategies of fishermen is diverse, spanning fisheries economics, anthropology and fisheries science. However two trends are clear: this work tends to either be general population-level explanations for how harvesting behavior may influence fish stock dynamics or specific individual-level case study descriptions of harvesting. Much of this work has been motivated by catch per unit effort (CPUE) measurements and their use in setting harvesting policies. 

\subsection{There is a literature suggesting how fleet behavior should change relationships between effort and catch}
Catch per unit effort is the amount of fish harvested by a fishermen per unit effort expended. Effort here typically refers to the amount of time spent with fishing nets in the water, but sometimes includes the time spent searching for fish \citep{HilbornWalters:1992}. It is the most commonly used metric for the analysis of fish stocks globally and is heavily relied upon in data-poor fisheries, as it's cheap and easy to collect \citep{Paulyetal:2013}.  Fisheries scientists traditionally assume that CPUE is proportional to the abundance of the stock. This has motivated considerable research into when that proportionality should be true \citep{@Harleyetal:2001}. In particular there has been a considerable amount of energy devoted to standardizing CPUE to account for increases in fishing power (improved nets, faster boats, better fish finders) but cannot be standardized to incorporate changes in fishing strategies such as sharing information with other boats \citep{Branchetal:2006}.  

Previous work has found that spatial characteristics, such as habitat and population patchiness, can alter the relationship between CPUE and stock abundance \citep{Stobartetal:2012, Walters:2003} and that depending on the information that fishermen have, that CPUE can remain stable much longer than abundance remains constant \citep{@Harleyetal:2001, Branchetal:2006}. These problems have been diagnosed both in the ecological and economics literature \citep{CookeBeddington:1984}. Further, this potentially major source of bias in fisheries management data has motivated a variety of other modeling work explicitly considering how human harvesting behavior can change the relationship between catch rates and abundance \citep{Hilborn:1985, DreyfusLeonKleiber:2001, HilbornWalters:1992}. Examples include both theoretical  and empirical work \citep{MangelClark:1983, HilbornLedbetter:1979} examining how fishermen allocate effort across fishing grounds, how catchability of a stock can change the relationship between catch and abundance \citep{ClarkMangel:1979}, what types of information fishermen respond to \citep{Vignaux:1996}, and how gear can change harvesting patterns \citep{Gaertneretal:1999}. This literature generally proposes explanations for fleet-level dynamics such as hyperdepletion or hyperstability in CPUE statistics. The explanations cover ecological characteristics, such as the spatial distribution of fish \citep{Hancocketal:1995}, and characteristics of the fishermen, such as their propensity for risk or efficiency of search and handling time, that tends to promote hyperstability or hyperdepletion \citep{Mahevasetal:2011, HilbornWalters:1992}. However this work makes no prediction about how harvesting pressures should change spatial distributions of catch and how this will affect harvesting strategy in the longer term \citep{HilbornWalters:1987}. 


\subsection{There is literature demonstrating what variables best predict fishermen behavior (effort allocation) in specific contexts and an individual-level}
Previous work on fisher decision making: "e.g. Jentoft 2000; Nielsen 2003; Hatcher and Pascoe 2006; Wilson et al. 2006; N�stbakken 2008" \citep{Fultonetal2011}

While much of the work examining fleet dynamics (i.e. population level mechanisms) is rooted in fisheries science, the literature detailing individual-level behavior is much more diverse. Both economists and anthropologists join fisheries scientists in considering the individual harvesting behavior of fishermen, if not for the same purposes. As outlined above, the fisheries science work typically covers fleet-level dynamics motivated by the relationship between harvesting and CPUE. Individual-level models in fisheries science tend to be agent-based models focused on understanding decision-making processes in order to predict the effects of management \citep{DreyfusLeon:1999, SoulieThebaud:2006, GaertnerDreyfusLeon:2004}. The anthropological literature is primarily descriptive in coverage. This work includes application of foraging theory to artisanal fisheries \citep{Aswani:1998, Begossi:1992, Abernethyetal:2007, Daw:2008}, social norms in harvesting practices \citep{Acheson:1975, HollandSutinen:2000}, and how social networks can influence fishing practices and success \citep{DurrenbergePalsson:nd, PalssonDurrenberger:1982, HilbornLedbetter:1979, CronaBodin:2006}. Anthropology, like much of the economic literature also considers choice of fishing location by fishermen \citep{Gatewood:1983, Orth:1987, Gezelius:2007}. 

The economic literature, however, tends to be much more statistical in nature, highlighting relationships between biophysical or economic characteristics and catch statistics. These studies tend to focus on modeling location choice of fishermen, conceptualizing fishing a set of discrete locations over which a fishermen chooses the option that gives him the highest utility \citep{Smith:2000}. These studies are short-term in scope, always considering behavior within a season and excluding how distributions of harvesting pressure may influence stock recruitment or distribution in subsequent seasons and all of the empirical work focuses on commercial fisheries. Some of these studies focus on whether fishermen are risk-seekers or risk-avoiders \citep{MistiaenStrand:2000}, but most are empirical studies using random-utility models to predict fishing choice based on logbook data \citep{BockstaelOpaluch:1983, EalesWilen:1986, Huttonetal:2004, Salasetal:2004, CurtisMcConnell:2004, EggertTveteras:2004, Vermardetal:2008, Andersenetal:2012, Marchaletal:2009, Valcic:2009, Eisenacketal:2006, Dupontetal:1993}. These random-utility models tend to find that weather, CPUE and distance are all significant predictors of location choice. 

\section{The calls for understanding fishermen behavior because it will improve predictions of management impacts but examples of how or what might have changed had we incorporated fishing behavior is hard to find. }

\subsection{Why management should change fishing behavior}
"All forms of management can create unintended consequences by modifying incentives of resource users (and other actors in the fishery) and hence their behaviour (Merton 1936). All actors in a fishery are driven by a mix of long and short-term drivers (Smith et al. 1999)." \citep{Fultonetal2011}

\section{Although there are a handful of examples that show how including fishermen behavior alters predicted outcomes of management. There is no way to generalize this outcome to understand when and how fishermen behavior should be important.}
The economic work has coupled random-choice models with age structured metapopulation models to understand how the system changes over the short term after management changes \citep{Smithetal:2008, Wilenetal:2002}. Others have been agent-based-models (ABMs) in which fishermen are given a set of rules for how to interact and allowed to move about freely \citep{HilbornWalters:1987, AllenMcGlade:1986, McGarvey:1994, Littleetal:2004, Wilsonetal:2007} And there are more general linked simulation models which allow the biology, economic and socio-cultural variables to vary endogenously \citep{Plaganyietal:2013} and theoretical work that models the linked dynamics of fish and management \citep{CarpenterBrock:2004, Crepin:2007, Kramer:2008}. There are even empirical examples \citep{BasurtoL2005, Basurto:2008, Stenecketal:2011}. These studies tend to find that incorporation of fishermen behavior causes significant deviations from traditional management analyses relying on the assumption that harvesting pressure is completely determined by management prescription. 

"Classical models predict that when sustained catches are taken from a multistock (or multispecies) fishery, those species with low growth rates will diminish towards commercial extinction (Ricker 1958; Paulik et al. 1967). However, these models assume spatial homogeneity in abun-dance and hence equal catchability among stocks (or spe-cies). In reality, each fishing opportunity has a particular mix of species, and placing catch limits on one species will merely shift fishing effort from fishing opportunities where that species is likely to be caught to fishing opportunities where it is less abundant. There is evidence for these shifts in the B.C. fishery, where total allowable catches were reduced for a few species (generally those with low productiv-ity) when ITQs were introduced. Catches of those species declined to match total allowable catches, but catches of other species in the fishery remained constant, suggesting that skippers were able to shift their fishing effort to alterna-tive fishing opportunities where other species dominated." \citep{Branchetal2005}

"For example, fishing seasons are a commonly used management tool, and reducing the length of the season is often used in an attempt to control effort (and hence limit catch). However, reducing season length does not usually reduce effort as much as anticipated, with effort becoming more temporally concentrated as fishers strive to maintain catch levels. At the extreme, this leads to situations such as the Bristol Bay red king crab fishery, which in 1996 lasted only four days and still exceeded its preseason guideline harvest level by 65\% (Briand et al. 2004). Spatial closures designed to protect vulnerable species, life stages or habitats can also have unintended consequences; effort displacement of the kind seen in the North Sea has led to increased pressure on other species or habitats not protected by the closures (Rijnsdorp et al. 2001; Dinmore et al. 2003)." \citep{Fultonetal2011}

"Output controls are not without problems either. Limited entry and quotas were introduced into the Bay of Fundy herring fishery in the 1970s to control fishing pressure, but instead they encouraged com-petition and capacity increases, which resulted in strong disparity in incomes (White and Mace 1988). In Iceland an unexpected consequence of quota introduction was the geographical concentration of quota ownership that had significant social and cultural implications (Helgason and Pa� lsson 1998). For southern bluefin tuna, the very high value of the fish combined with the difficulty of monitoring and enforcing regulations have led to catches way in excess of quota, with high rates of unreported fishing which in turn have increased uncertainty in assessments of resource status (Kolody et al. 2008; Polacheck and Davies 2008)." \citep{Fultonetal2011}.

"Fishers are also extremely good at maximizing opportunities, sometimes including exploiting loop-holes (Rowling et al. 1994). Mismatches between state and federal fisheries regulations in the US and Australia have led to fishers misreporting catches into areas with less restrictive catch limits (Larcombe et al. 2001; Branch et al. 2006). Unfor-tunately, unexpected responses to closure of appar-ent jurisdictional loopholes can also occur. Prior to 1990, the Norwegian small boat fishery was not part of the regulated trawl fishery, and was essen-tially self regulating. In 1990 a crisis in the Norwegian cod fishery meant that regulation was extended to encompass the small boats, creating a race to fish and incentives to fish more heavily (and invest in larger vessels) to secure access to the new and future fishing rights. The combined effect of the unpredicted response and the manner in which the regulation was implemented resulted in increased pressure on the cod (and other species) as effort increased rather than declined - an unintended consequence of introducing formal management measures (Maurstad 2000)." \citep{Fultonetal2011}

\section{Here we attempt a general synthesis that examines whether effort allocation varies predictably across fishery systems and under what conditions this behavioral change should have implications for management success.} 
What is missing from these studies is a general explanation of how individual foraging strategies will vary with characteristics of both the targeted fish and human harvesters and how this manifests as emergent population-level dynamics. This requires considering how social/economic drivers interact with ecological ones. This has been broadly recognized in the fisheries science community for decades \citep{Smith:2000, Graftonetal:2006, Arlinghausetal:2013, Fultonetal:2011, vanPuttenetal:2011, HayniePfeiffer:2012, DegnbolMcCay:2007, Hilborn:2007, Branchetal:2006, Hilbornetal:2005, Salasetal:2004, Wilenetal:2002}, although there is a particular push for understanding these coupled-dynamics between fishermen and fish with the popularity of ecosystem-based management (EBM) for marine systems. However a small subset of work has attempted to cross scales, and I review these studies here.  This is not the norm for studies targeting fisheries systems, and there's no unique trend for which systems are most widely represented. These studies are anthropological and focusing on artisanal fisheries \citep{BasurtoL2005, Basurto:2008}, economic focusing on commercial fisheries \citep{Smithetal:2008, SmithWilen:2003}, and the spectrum in between. 

\subsection{It's important not to just extrapolate statisticaly into the future}
"In combination these examples serve as a caution about simply extrapolating past fishing behaviour under changed management or stock status conditions." \citep{Fultonetal2011}/ 

\subsection{Previous work on generality of fishing behavior}
Common knowledge that gear affects location choice of fishermen: "Trawl skippers usually choose between known fishing opportunities, which are observed as large groups of trawls that are conducted in the same portion of a fishing ground, or go exploratory fishing" \citep{Branchetal2005}. 

"In the B.C. fishery, total catches are limited by quotas un-der the ITQ program. Vessel owners should therefore seek to maximize their profits by exactly catching their quota hold-ings, reducing costs, and landing species when market prices peak. Knowledge about species distribution and abundance in different fishing opportunities accordingly gains great value. Our analysis of the B.C. fishery shows that the aver-age skipper trawled in 26 different fishing opportunities (�15 trawls in each), although this varied greatly (2-69) among the top 70 vessels. Skippers in this fishery visit substantially more locations than were reported for Icelandic skippers fishing cod, who generally visited only a few locations and were considered to be not very innovative (Durrenberger and P�lsson 1986). Their study focused on locations that were 225 km2 but noted that each location might contain a num-ber of small �fishing spots�, which would more likely corre-spond to our fishing opportunities. The great number of target species in the B.C. fishery probably also motivates skippers to fish in a wide variety of fishing opportunities.

\section{Fishery systems vary in 4 ways: the ecology/ies)  of the species targeted, the type of management, the type of gear used, and the type of information to which a fishermen has access.}

\section{We present a review of the work that has focused on fishing behavior: location choice models. These are examples of studies that attempt to predict where and when fishermen should fish. This is the foundation of the type of behavior most calls are for.}

\section{We record what the characteristics of the fishery are (ecology, management, gear, information) and what factors are found to be important. We then examine whether the factors that are important vary in any meaningful way. And try to explain that variation.}


\end{document}  
\documentclass[12pt,a4paper]{report}
\let\oldmarginpar\marginpar
\renewcommand\marginpar[1]{\-\oldmarginpar[\raggedleft\footnotesize #1]%
{\raggedright\footnotesize #1}}
%\usepackage{charter}
\usepackage{graphicx}
\usepackage{amsmath}
\usepackage{amssymb}
\usepackage{mathtools}
\usepackage{array}
%\usepackage[bitstream-charter]{mathdesign}	% using charter font for math
\usepackage{natbib}
\usepackage{gensymb}
\usepackage{epstopdf}
\usepackage{multirow} %for multi column spanning rows or multi row spanning columns
 \usepackage{rotating} %for rotating tables
 \usepackage{subfig} %to have figures within figures
\usepackage{multicol}
 \usepackage{ulem} % Allows for underlining without weird formatting glitches. 
 \usepackage{pifont}% Allows for use of ZipfDingbats font
\usepackage[usenames,dvipsnames,svgnames,table]{xcolor} % Allows colored font
\usepackage{hyperref} %allows hyperlinks in the document
\usepackage[linecolor=black,backgroundcolor=white,bordercolor=white,textsize=small]{todonotes}
\usepackage{wrapfig}
\usepackage[top=1in, bottom=1in, outer=1in, inner=1in, heightrounded, marginparwidth=1in, marginparsep=0.25in]{geometry}
\usepackage{marginnote}
\usepackage{setspace}
\usepackage{hyperref}
\usepackage{makeidx}	% package for building and including an index
\makeindex	% makes the index
%\usepackage{fontspec} % Provide features for AAT and OpenType fonts
%\setmainfont{Helvetica Light} % Define the default font family
\usepackage{lineno}

\usepackage[T1]{fontenc}
\usepackage[scaled]{helvet}

%\usepackage{tgheros}
%\renewcommand*\familydefault{\sfdefault} %% Only if the base font of the document is to be sans serif
%\usepackage[T1]{fontenc}


\renewcommand*\familydefault{\sfdefault} %% Only if the base font of the document is to be sans serif


\title{{\bf Individual behavior and emergent community dynamics in marine systems}\\
{\Large Improving fisheries management}} 
\date{}
\author{Literature Review and Thesis Proposal for the Generals Exam \\
Emma Fuller \\
\vspace{4cm}\\
May 22, 2013 \\
\vspace{5cm}\\
Advisor: Simon Levin\\
Generals Committee Chair: Steve Pacala \\
Additional Committee Members: David Wilcove, Rob Pringle \\
\vspace{1.5cm} \\
Department of Ecology \& Evolutionary Biology \\
Princeton University}
\begin{document}

\maketitle
\onehalfspacing
%\linenumbers
\renewcommand{\abstractname}{Overview}
\begin{abstract}
My proposed research goal is to determine how harvesting strategies at the level of individual fishermen should scale up to affect population level processes, and specifically how fishermen should change their harvesting behavior when a marine protected area (MPA) is established. The motivation for this work comes from multiple scales. At the broadest and most general level this work will help to determine how linkages between people and the resources they use affect the complex adaptive systems of which both human and ecological systems form a part. At a medium scale this proposed work will help determine how feedbacks between individual behavior and population dynamics shape fisheries systems, and at the finest scale this proposed work will determine how ecological and social dynamics intersect to shape where and when fishermen harvest fish. 

In the first chapter I will review the literature to give a background on how harvesting strategies have been considered in the fisheries literature, focusing on efforts to understand and incorporate fleet-level dynamics into traditional stock-assessment models and specific case studies that examine the fishery dynamics at the level of the individual. This review sets the stage for an understanding of what's lacking in fisheries modeling: a connection between individual level behavior and fleet (population) level processes. I will then turn to the ecological literature and discuss how animal foraging strategies have been studied, highlighting optimal and social foraging theory, game theory along with the concept of an evolutionarily stable strategy (ESS), and individual-level attempts to explain community level dynamics. Finally I will briefly outline that while both fisheries science focused on harvesting behavior and ecology work at both the individual and population level, drawing them together to consider the feedbacks between scales is still an area where much work is left to be done from a theoretical viewpoint, and an area that may inform policy and management. 

In the second chapter I will cover my proposed work. I first introduce and motivate my set of questions. The first section describes a nonlinear functional response that incorporates grouping by fishermen and their targeted stocks and examines how this may change system stability and dynamics. The second section covers the determination of how individual harvesting strategies should vary according to fishery system characteristics, which requires both theoretical and empirical work. The third section details how the composition of a fleet may affect the optimal harvesting strategy. The fourth section asks: do artisanal fisheries respond different to a new MPA than commercial fisheries?
\end{abstract}

\tableofcontents

\chapter{Literature Review}
From an ecological perspective fishermen are the top predators of their system, though their payoff is measured in dollars rather than calories. In the following literature review I will cover how fishermen's harvesting behavior has been covered in fisheries science and how predator-prey theory has quantified and defined foraging behaviors more generally in consumer-resource interactions. 

\section{Harvesting strategies in the fishing literature}
The literature covering harvesting strategies of fishermen is diverse, spanning fisheries economics, anthropology and fisheries science. However two trends are clear: this work tends to either be general population-level explanations for how harvesting behavior may influence fish stock dynamics or specific individual-level case study descriptions of harvesting. Much of this work has been motivated by catch per unit effort (CPUE) measurements and their use in setting harvesting policies. 

Catch per unit effort is the amount of fish harvested by a fishermen per unit effort expended. Effort here typically refers to the amount of time spent with fishing nets in the water, but sometimes includes the time spent searching for fish \citep{HilbornWalters:1992}. It is the most commonly used metric for the analysis of fish stocks globally and is heavily relied upon in data-poor fisheries, as it's cheap and easy to collect \citep{Paulyetal:2013}.  Fisheries scientists traditionally assume that CPUE is proportional to the abundance of the stock. This has motivated considerable research into when that proportionality should be true \citep{@Harleyetal:2001}. In particular there has been a considerable amount of energy devoted to standardizing CPUE to account for increases in fishing power (improved nets, faster boats, better fish finders) but cannot be standardized to incorporate changes in fishing strategies such as sharing information with other boats \citep{Branchetal:2006}.  

Previous work has found that spatial characteristics, such as habitat and population patchiness, can alter the relationship between CPUE and stock abundance \citep{Stobartetal:2012, Walters:2003} and that depending on the information that fishermen have, that CPUE can remain stable much longer than abundance remains constant \citep{@Harleyetal:2001, Branchetal:2006}. These problems have been diagnosed both in the ecological and economics literature \citep{CookeBeddington:1984}. Further, this potentially major source of bias in fisheries management data has motivated a variety of other modeling work explicitly considering how human harvesting behavior can change the relationship between catch rates and abundance \citep{Hilborn:1985, DreyfusLeonKleiber:2001, HilbornWalters:1992}. Examples include both theoretical  and empirical work \citep{MangelClark:1983, HilbornLedbetter:1979} examining how fishermen allocate effort across fishing grounds, how catchability of a stock can change the relationship between catch and abundance \citep{ClarkMangel:1979}, what types of information fishermen respond to \citep{Vignaux:1996}, and how gear can change harvesting patterns \citep{Gaertneretal:1999}. This literature generally proposes explanations for fleet-level dynamics such as hyperdepletion or hyperstability in CPUE statistics. The explanations cover ecological characteristics, such as the spatial distribution of fish \citep{Hancocketal:1995}, and characteristics of the fishermen, such as their propensity for risk or efficiency of search and handling time, that tends to promote hyperstability or hyperdepletion \citep{Mahevasetal:2011, HilbornWalters:1992}. However this work makes no prediction about how harvesting pressures should change spatial distributions of catch and how this will affect harvesting strategy in the longer term \citep{HilbornWalters:1987}. 

While much of the work examining fleet dynamics (i.e. population level mechanisms) is rooted in fisheries science, the literature detailing individual-level behavior is much more diverse. Both economists and anthropologists join fisheries scientists in considering the individual harvesting behavior of fishermen, if not for the same purposes. As outlined above, the fisheries science work typically covers fleet-level dynamics motivated by the relationship between harvesting and CPUE. Individual-level models in fisheries science tend to be agent-based models focused on understanding decision-making processes in order to predict the effects of management \citep{DreyfusLeon:1999, SoulieThebaud:2006, GaertnerDreyfusLeon:2004}. The anthropological literature is primarily descriptive in coverage. This work includes application of foraging theory to artisanal fisheries \citep{Aswani:1998, Begossi:1992, Abernethyetal:2007, Daw:2008}, social norms in harvesting practices \citep{Acheson:1975, HollandSutinen:2000}, and how social networks can influence fishing practices and success \citep{DurrenbergePalsson:nd, PalssonDurrenberger:1982, HilbornLedbetter:1979, CronaBodin:2006}. Anthropology, like much of the economic literature also considers choice of fishing location by fishermen \citep{Gatewood:1983, Orth:1987, Gezelius:2007}. 

The economic literature, however, tends to be much more statistical in nature, highlighting relationships between biophysical or economic characteristics and catch statistics. These studies tend to focus on modeling location choice of fishermen, conceptualizing fishing a set of discrete locations over which a fishermen chooses the option that gives him the highest utility \citep{Smith:2000}. These studies are short-term in scope, always considering behavior within a season and excluding how distributions of harvesting pressure may influence stock recruitment or distribution in subsequent seasons and all of the empirical work focuses on commercial fisheries. Some of these studies focus on whether fishermen are risk-seekers or risk-avoiders \citep{MistiaenStrand:2000}, but most are empirical studies using random-utility models to predict fishing choice based on logbook data \citep{BockstaelOpaluch:1983, EalesWilen:1986, Huttonetal:2004, Salasetal:2004, CurtisMcConnell:2004, EggertTveteras:2004, Vermardetal:2008, Andersenetal:2012, Marchaletal:2009, Valcic:2009, Eisenacketal:2006, Dupontetal:1993}. These random-utility models tend to find that weather, CPUE and distance are all significant predictors of location choice. 

What is missing from these studies is a general explanation of how individual foraging strategies will vary with characteristics of both the targeted fish and human harvesters and how this manifests as emergent population-level dynamics. This requires considering how social/economic drivers interact with ecological ones. This has been broadly recognized in the fisheries science community for decades \citep{Smith:2000, Graftonetal:2006, Arlinghausetal:2013, Fultonetal:2011, vanPuttenetal:2011, HayniePfeiffer:2012, DegnbolMcCay:2007, Hilborn:2007, Branchetal:2006, Hilbornetal:2005, Salasetal:2004, Wilenetal:2002}, although there is a particular push for understanding these coupled-dynamics between fishermen and fish with the popularity of ecosystem-based management (EBM) for marine systems. However a small subset of work has attempted to cross scales, and I review these studies here.  This is not the norm for studies targeting fisheries systems, and there's no unique trend for which systems are most widely represented. These studies are anthropological and focusing on artisanal fisheries \citep{BasurtoL2005, Basurto:2008}, economic focusing on commercial fisheries \citep{Smithetal:2008, SmithWilen:2003}, and the spectrum in between. 

The economic work has coupled random-choice models with age structured metapopulation models to understand how the system changes over the short term after management changes \citep{Smithetal:2008, Wilenetal:2002}. Others have been agent-based-models (ABMs) in which fishermen are given a set of rules for how to interact and allowed to move about freely \citep{HilbornWalters:1987, AllenMcGlade:1986, McGarvey:1994, Littleetal:2004, Wilsonetal:2007} And there are more general linked simulation models which allow the biology, economic and socio-cultural variables to vary endogenously \citep{Plaganyietal:2013} and theoretical work that models the linked dynamics of fish and management \citep{CarpenterBrock:2004, Crepin:2007, Kramer:2008}. There are even empirical examples \citep{BasurtoL2005, Basurto:2008, Stenecketal:2011}. These studies tend to find that incorporation of fishermen behavior causes significant deviations from traditional management analyses relying on the assumption that harvesting pressure is completely determined by management prescription. 

\section{Foraging strategies in the ecological literature}
There is a long history of work that examines how individuals should allocate effort to find food in space and time. This work starts with optimal foraging theory (OFT). OFT begins from the assumption that consumers will forage in such a way as to maximize their energy intake per unit time \citep{Krivan:2010}. This field began with classic work by \cite{MacArthurPianka:1966} considering how individuals should forage in a heterogenous environment, \cite{Charnov:1976} who extended the diet choice models and first proposed the Marginal Value Theorem (MVT), and Stephens and Krebs (1986) who codified the field. 

The questions optimal foraging theory addressed were at the level of individual animal behavior and static concepts. Patch choice models, as begun by \cite{MacArthurPianka:1966}, considered if patches are listed according to quality, in descending order, how many patches should a predator optimally visit (gains in calories greater than the loss). The two variables considered are the average time spending hunting for prey within a patch and the mean travel time (between patches) per prey item eaten.

The optimal diet model, introduced in the same paper, is similar: if prey items are ranked from best to worst, the questions is how many of them should be included in the diet while a predator is searching within a patch. The variables considered are the time it takes to search for a prey item and the time it takes to pursue it. Adding more prey decreases the time it takes to search (assuming prey encountered randomly, in proportion to abundance in patch), but will increase mean pursuit time. However this work required strict simplifying assumptions: individuals did not interact with either each other, searching was random, resources were typically not able to be depleted in a dynamic way, and environments were homogenous. In short, no population-level conclusions were able to be made. 

The extension to consider interactions between individuals primarily characterizes social foraging theory. This work explicitly incorporates a game theoretic approach in which payoffs for individual behavioral strategies are now dependent on not just the consumer itself, but by others in the system. Major developments in social foraging theory include the formulation of producer-scrounger games, and updating classical optimization assumptions \citep{GiraldeauCaraco:2000}. 

Indeed, this inclusion of game theory in behavioral models brings Maynard Smith's concept of an Evolutionarily Stable Strategy (ESS) back to the fore and suggests inclusion of longer timescales. An ESS is a strategy such that no other strategy could invade the population and receive a better payoff then the one currently resident \citep{MaynardSmith:1974}. This takes into account not only the individual's characteristics in determining optimal behavior, but takes into account how others' behaviors will influence your own payoff, as is often the case in predator-prey or competitive interactions. 

Individual interactions are the foundation for emergent population- and community-level dynamics such as density dependence and species interactions, but individual-level work has had relatively little exchange with population and community ecology. Both population and community ecology consider dynamics over time, and how populations responded to perturbations and changes in their environment, biotic or abiotic. Population biology resolves at the level of the individual and, unlike work on foraging behavior, social or not, considers feedbacks at ecological time scales (density-dependent interactions specifically). Community ecology extends this to consider implications of heterogeneous individuals (i.e. multiple species) interacting with one another. 

Predator-prey interactions, a subset of community ecology,  started considerably earlier than foraging theory with Lotka and Volterra's independent derivation of the now canonical Lotka-Volterra predator prey model \citep{Volterra:1928, Lotka:1925}. This model was significant for its oscillatory behavior but had trivially simple assumptions, such as that the encounter rate of predators and their prey good be based of mean-field approximations \citep{Lawetal:2000, Gurarieetal:2013}. This simplicity motivated updates, first by Holling, to develop alternative functional responses \citep{Holling:1959}. Through a series of lab experiments, field manipulations and thought experiments Holling developed three types of functional responses and introduced encounter rates and handling times as important behaviors to consider in these interactions. A functional response determines the per-capita predation rate for a given abundance of prey. Type-I is the traditional linear relationship, predation rate increases linearly as prey abundance increases, so predation rate could continue (theoretically) to infinity.  Type-II functional response is saturating at some density of prey. The presumed mechanism here is that handling time limits the predator's intake rate of prey at some high density, as search time approaches zero and encounters become near instantaneous. The final functional response is Type-III and is sigmoidal. This means that at low prey densities a predator's intake rate of prey is low, but at some intermediate density increases nonlinearly until saturating like the Type-II functional response. The nonlinear increase in intake rate with increasing prey abundance has a number of mechanisms proposed for it, including learning on the part of the predator or spatial refuges for prey in the habitat (see figure \ref{functional response}). 

\begin{figure}
\centering
\includegraphics[width=1\textwidth]{functional_response.pdf}
\caption{\small Predator's per capita consumption rate of prey as prey density increases. Per capita consumption rate (C) for predators are shown for Type I, Type II and Type III functional responses, from left to right. r can be thought of as risk of predation, h is handling time. Adapted from \cite{DennoLewis:2009}. \label{functional response}}
\end{figure}

Functional responses represent the closest population and community ecology work has come to incorporating individual level heterogeneity into population level processes \citep{Fryxelletal:2007}. Despite individual- and population/community-level work developing relatively separately, there is a growing sense that only when these two scales are combined will we begin to get a better sense of how communities respond to variation in the environment \citep{Bolnicketal:2011}.

\section{Social Ecological Systems}
In this review I have focused on how rare it is for studies to consider how economic and ecological drivers interact to influence harvesting patterns, which feed back via profit to the economic system and biomass to the ecological one. The newly emerging focus on social-ecological systems (SES) has a great deal of conceptual insight to offer this focus. Linking both social and ecological processes allow one to consider human influenced systems on a longer time scale \citep{Schluteretal:2012, Ostrom:2009, Liuetal:2007, Liuetal:2007b}. However, even here many models fail to represent social-ecological systems in a dynamic way, and this echoes the need of foraging-type models to be able to incorporate across scales \citep{Janssenetal:2000}. Central to the study of social-ecological systems is the investigation of the structure and dynamics of feedbacks within and between the ecological and social subsystems  \citep{MacyWiller:2002, JeffreyMcintosh:2006}. These have been suggested to have significant effects on the existence of multiple stable states and critical transitions (Lade et al. (in prep)). Fisheries are a model social ecological system as the populations of fish and the humans that harvest them are individually well studied \citep{Branchetal:2006}. This makes the linkages between human behavior and ecological dynamics easier to isolate, and MPAs make a nice case study of a perturbation to that system. 

\subsection{Marine Protected Areas as a Case Study}
Marine protected areas (MPAs) are gaining popularity as a management tool \citep{Halpernetal:2012}. These protected areas typically limit or prevent extractive human activity within them, including fishing. From a conservation standpoint, marine protected areas function very like their terrestrial counterparts: by preventing human activity they allow animals vulnerable to disturbance a space to grow. A marine protected area can benefit a fishery in which it's placed by providing larval export and also through spillover of adult fish into areas where fishing is allowed. MPAs are recommended primarily on the basis of these ecological characteristics \citep{GellRoberts:2003} along with insurance for variable population recruitment \citep{KahuiAlexander:2008,Gainesetal:2010}. These arguments typically only account for the assumed dispersal network present \citep{Sanchirico:2004, Sanchirico:2005}, rarely taking species interactions or fishing practices into account.

Like any management tool, the success of a given marine protected area must be judged relative to the goals at establishment. As MPAs have been evaluated, trajectories have been mixed. Generally, more fish biomass has been found inside the reserves than outside of them \citep{PoluninRoberts:1993, RussAlcala:1996, McClanahanMangi:2000, Halpern:2003, Lesteretal:2009}.  MPA reviews have typically focused on conservation benefits, thus studies that examine the recovery of fish stocks outside the reserve are much fewer. Analysis of MPAs set up to provide benefits for fisheries have less clear results; it's hard to see evidence that spillover matches or improves upon lost yield in fisheries \citep{Hilborn:OceanCoastalManagement:2004}, although some work does exist that demonstrates increases in catch \citep{Russetal:2004, JanuchowskiHartleyetal:2012, Alemanyetal:2013} and recruitment \citep{Harrisonetal:2012}. There are some economic analyses of benefits to fisheries from MPAs, but these tend to find weak or non-existent benefits accruing to fishermen \citep{Smithetal:2006, SanchiricoWilen:2001, Hannesson:1998, HollandBrazee:1996}. This may be because MPAs assume that effort is displaced out of the fishery or historical effort remains the same, which is not always case \citep{Branchetal:2006}. 

There is a subset of work that considers how establishing marine protected areas  will displace fishing effort, although notably these are not the same studies that advocate for MPA establishment. These studies tend to be short term analyses of how establishing a reserve will change the ecology of the targeted species \citep{Dinmoreetal:2003}, economic analyses of how fishermen will change how they choose locations to fish based on the changed distribution of available patches \citep{SanchiricoWilen:2001, Wilenetal:2002, SmithWilen:2003, Sanchiricoetal:2006, ValderramaAnderson:2007, KahuiAlexander:2008}, or simulation models \citep{Littleetal:2005}. There is, however, no empirical work quantifying how these harvesting behaviors change.  Further, much of the reviewed work are simulations with arbitrary rules for interactions among fishermen, although many parameterize simulations based on existing catch data. While most of these studies consider feedbacks between the ecological dynamics of the targeted stock and harvesting behavior, rarely do they consider how strategies change within a season as fish abundances become locally depleted. 

I posit that by understanding the mechanistic basis for fishermen's distribution of effort across space, I can examine how these distributions of effort should change with establishment of marine protected areas in a general framework that's not limited to a single case study. This approach requires integrating from the level of the individual fishermen to that of the fleet, as management regulates fleet-level dynamics. However these fleet level dynamics are based off both ecological and economic drivers. Thus understanding how these economic and ecological drivers intersect to shape harvesting strategies provide a way to examine the long term stability and sustainability of fishery systems.

\chapter{Proposal}
The goal of this proposal is to understand how harvesting strategies of fishermen vary with system characteristics and scale up to affect fleet and fish-population dynamics. This work will be both theoretical and empirical in nature, incorporating analytical and agent-based modeling, and analysis of catch data in US commercial and, ideally, artisanal fisheries. This work will focus specifically on the management tool of marine protected areas (MPAs). 

Fisheries are examples of social-ecological systems, systems which have human and ecological sub-systems linked by processes (i.e. fishing) that influence dynamics in both. In fact, fisheries can be thought of, and are typically managed as, a density dependent predator-prey system. In the United States fisheries managers have traditionally evaluated stock abundance over a large area and set the total allowed harvest for each year. This type of management conforms to mean-field assumptions: that the variation across space among individual fishermen and fish will average out. Thus, at a fleet- and stock-level, management can comfortably ignore variations across space. 

As management becomes increasingly spatial, especially with the increased excitement surrounding marine protected areas, there is a mismatch between the way fisheries managers think of and model the system and the type of spatial dynamics MPAs require consideration of. Indeed this mismatch is readily apparent in the way that fishermen's behavior is incorporated in modeling of MPAs. It is commonly accepted that success of MPAs depend on life history characteristics of fish, however it also depends on spatial distribution of mortality (i.e. fishing) \citep{Halpernetal:2004}. But models of MPAs tend to assume constant effort (so effort is proportionally reduced by area closure) or effort is uniformly redistributed through system after MPA establishment \citep{Branchetal:2006}.

Further, meta-analytic work searching for socio-economic variables that predict MPA success found complicated relationships. Results indicate that relationships might be region specific \citep{Pollnacetal:2010}. Given that models simulating MPAs tend to ignore fishermen behavior, is it possible that harvesting strategies could provide a general explanation for these seemingly case-specific results without having to resort to more qualitative social interpretation? This leads me to four specific questions:

\begin{enumerate}
\item How do the complex spatial dynamics of fish and fisherman affect the performance of traditional fisheries management?
\item Does adaptive behavior in fishermen affect the efficacy of an MPA?
\item Does the composition of the fleet (i.e. single vessels vs multi-vessel firms, etc.) affect how strategy should change in response to an MPA?
\item Are artisanal fisheries different than commercial in their response to a new MPA?
\end{enumerate}

\section{How do the complex spatial dynamics of fish and fisherman affect the performance of traditional fisheries management?}
\subsection{Introduction}
Fisheries science is focused on understanding and managing fisheries, and hence deals with the intersection of ecological and economic forces and dynamics. Starting with the Gordon-Schaefer model \citep{Schaefer:1954} there has been a tradition of modeling dynamics of effort within fisheries and harvested fish stocks to make predictions for how these two forces play out. The Gordon-Schaefer model made clear predictions that an open access fishery could expect a tragedy of the commons to occur. In this model fish abundance N grew logistically and fish were harvested with constant effort $E$ as follows,

\begin{equation}
\frac{dN}{dt} = rN\left(1-\frac{N}{K}\right)-qEN,
\end{equation}

\noindent where $r$ is the intrinsic growth rate of the fish, $K$ is the carrying capacity, and $q$ is the catchability coefficient ($< 1$). From this equation and assuming a fixed price, maximum sustainable yield (MSY) can be calculated and thus an optimal harvest rate can be assigned. 

Since this classic model, theoretical treatments of the dynamics of fisheries have become increasingly ecologically and economically sophisticated. Ecologically the Gordon-Schaefer model of a single uniform fish stock has given way to size- and age-structured fish populations, and then from single species to multiple species treatments \citep{May:Science:1979} to ecosystem-based management approaches \citep{Hilborn:2012}. Developments in fisheries economics have relaxed the requirement of constant effort, constant price and costs, and others \citep{Smith:2012}. Despite the considerable advances in sophistication and detail in these models, few studies consider how the ecology of the fish and the behavior of the fishermen jointly determine a fishery system's dynamics. The Schaefer model and the more ecological and economically sophisticated models hence have all kept the same assumption: that catch-per-unit-effort (CPUE) relationship to stock abundance remains fixed. 

Fisheries systems are inherently a predator-prey system, with humans playing the predator. Similar to Lotka and Volterra's predator-prey model, effort is assumed constant in the Gordon-Schaefer model, and hence catch per unit effort is expected to vary linearly with prey abundance. This same assumption is found in the Lotka-Volterra predator prey model, and is based on assumptions rooted in the analysis of ideal-free-gases \citep{HutchinsonWaser:2007, Gurarieetal:2013}. To relax this assumption in predator-prey models, \citep{Holling:1959} introduced the idea of functional responses. These functional responses can account for nonlinear interactions between predators and prey and have been used to account for mechanisms such as handling time, learning by predators and prey-switching \citep{Anderson:2010}. 

Indeed there is some work that draws connections between predator-prey modeling and fisheries systems. \cite{AllenMcGlade:1986} developed a set of models based on Lotka-Volterra predator-prey models, and explicitly recognized fishing as taking place within a complex system that is shaped both by biophysical and human dynamics. That said, there are a number of case studies that examine fishermen's harvesting behavior over space, but none that consider long term feedbacks between the two subsystems. Empirically there are examples of studies incorporating nonlinear responses of fishers to changes in fish abundance. \cite{Carpenteretal:1994} used a 33 year data set of anglers and fish in a Wisconsin freshwater lake. This study found that fish had density-dependent dynamics but that harvesting by anglers followed a type III functional response. \cite{Lorenzenetal:2006} estimated Yield-Effort curves for multiple species on a reef and concluded that the Schaefer model is not supported, while a sigmoidal response is. 

There has been a general treatment of under what conditions the linear response between abundance and CPUE is appropriate. However there has been no general analytic analysis of what a nonlinear CPUE (functional response) does to typical management reference points such as MSY or their stability. There are a number of instances of studies looking at optimal harvesting rates of interacting species, but no simple analytical studies that consider that the harvesting behavior itself may vary nonlinearly with fish abundance. 

\subsection{Proposed Methods}
As a first step in considering how incorporation of fishermen harvesting behavior can change ecological dynamics, I will analytically determine how changing a fisherman's functional response changes the dynamics of a fishery system. Here fishermen will have a type III functional response to fish rather than a linear one. Examining this harvesting equation graphically, the possibility of multiple stable states is immediately clear.

\begin{figure}
\centering
\includegraphics[width=0.75\textwidth]{type3.pdf}
\caption{\small Plotting the growth rate of the fish stock in blue and two functional responses in grey (Holling type-I functional response) and red (Holling type-III functional response). Where the growth curve crosses the functional response the system is at equilibrium. \label{type3}}
\end{figure}

In figure \ref{type3} the blue and grey lines correspond to the growth and harvest rate in the Gordon-Scheffer model, respectively. Where the two intersect is where the system is at equilibrium. The red line represents harvesting if fishermen had a Holling type III functional response. By changing the way that fishermen harvest fish, multiple stable states may be introduced. Here I propose to analyze a type III functional response and find MSY and compare it to the same system with the traditional linear harvesting response. 

While it is graphically clear that a Holling type III functional response can introduce multiple equilibria, it's not clear what mechanism drives the nonlinear increase in harvest per unit effort as abundance of fish grows.  Previous work by \cite{Fryxelletal:2007} has shown that grouping of predators and prey can change the functional response by altering handling times and encounter rates. Using these functional responses as a starting point and discussion with James Watson, Andrew Tilman, Eleanor Brush, and Adrienne Tecza, I have developed a model featuring nonlinear harvesting response based on fishermen grouping in space. 

This functional response approximates whether fishermen follow each other and examines fleet predation rate. The parameters are as follows: 

\begin{table}[htdp]
\caption{\small Parameters for developed functional response}
\begin{center}
\begin{tabular}{cll}
\hline
\hline
\bf Parameters & \bf Definition & \bf Dimension\\
\hline
\hline
$\omega$ & Speed of fishing vessel & Meters/time \\
$\beta$ & Width of fish-finder & meters \\
$\alpha$ & Spatial decorrelation & Dimensionless \\
$h_1$ & Time taken to haul fish onto boat & Time/fish \\
$h_2$ & Time taken to process fish on boat & Time/fish \\
$T$ & Total number of fishermen & dimensionless \\
$q$ & Catchability & dimensionless \\
\hline
\end{tabular}
\end{center}
\label{default}
\end{table}%

\noindent So the search rate becomes $a = \omega\beta q$, which has units of $\frac{m^2}{\text{time}}$. The fleet level predation rate (functional response) is then 

\begin{equation}
\frac{\alpha a T N}{1+a(h_1+\alpha h_2)N},
\end{equation}

\noindent where $N$ is the density of fish. Similar to \cite{Fryxelletal:2007} I incorporate grouping on the part of the fish and the functional response becomes in the full model 

\begin{equation}
\frac{dN}{dt} = rN\left(1-\frac{N}{K}\right)-\frac{\alpha a T N^b}{1+a(h_1+\alpha h_2)N^b},
\end{equation}
\noindent where $b<1$. Therefore with this functional response the fisherman's per capita encounter rate decreases with increased overlap, as $\alpha$ goes to $\frac{1}{T}$.  This approximates the overlap in the each fisherman's visual field as they draw closer together. 

With this functional response I'd like to look at how this changes the traditional management benchmark of maximum sustainable yield, and under what conditions this functional response introduces multiple stable states. To calculate MSY I consider $\omega\beta$ (the search rate) as the equivalent of effort $E$. Thus I need to consider how the yield is maximized in relation to the search rate. Finally I consider how profit ($\pi$) must change. Change in profit can be represented as 

\begin{equation}
\frac{d\pi}{dt} = \hat{p}\left[\frac{\alpha aN^b}{1+a(h_1+\alpha h_2)N^b}\right]-c(\omega\beta),
\end{equation}

\noindent where $\hat{p}$ is the price per unit biomass and $c$ is the cost per unit effort (meters$^2$). Considering profit allows me to incorporate command and control-type management (top-down) versus cooperative (bottom-up) management. It's possible that bottom up management approaches would affect the encounter rate via information sharing, while top down might operate through the cost function. 

\subsection{Expected Results}
The result will likely be that fishermen having a type III functional response introduces multiple stable states into the dynamics. Thus human harvesting allows the possibility of critical transitions that were not present when fishing did not occur. Further I expect that these transitions between one stable state and another will be governed by the parameter values describing the functional response of fishermen and growth rate of the stock. Considering how these parameters change the landscape of stability may provide some insights into the way that a change in system characteristics can lead a previously stable fishery to collapse without having to increase harvest rates.  This may be important to management if fishermen do indeed exhibit a type III functional response. Finally, it may be possible to consider how the spatial overlap of fishermen in space ($\alpha$) evolves over time using adaptive dynamics. 

I expect this work to provide evidence that how fishermen respond to fish abundance affects the stability of the entire system, however this work is fundamentally phenomenological. There is no guarantee that fishermen should be more or less correlated in space. To understand whether these non-linear functional responses are likely to occur in reality, I need to better understand what types of individual incentives and harvesting behavior can lead to this type of functional response at the fleet level. 
\section{Does adaptive behavior in fishermen affect the efficacy of an MPA?}

\subsection{Introduction}
To answer the linked questions of how the harvesting behavior of fishermen changes after MPA establishment and whether these new harvesting strategies affect the dynamics and stability of the linked fish-fishermen system, I will make use of the extensive literature on predator-prey metapopulation models. In these models the dynamics and their stability depend on the movement rules allowing predators (fishermen) and prey (fish) to move between patches, dynamics of prey, time-scale separation, and the functional responses of predators (with particular attention to encounter rates) \citep{Bernsteinetal:1999}. Using this modeling framework allows me to focus on how traits of individual fishermen and the ecology of the target fish change the fishermen's functional response and movement rules, and therefore the stability of the entire system. 

The goal is to understand how these functional responses change depending on individual harvesting behavior. This will allow me to choose the model that will best fit a given fishery's characteristics. I want to understand what drives harvesting strategy sufficiently such that I can vary a predator-prey metapopulation model to reflect changes in individual behavior of fishermen and draw connections between individual level behavior and community level, fishery dynamics. To be able to apply this modeling framework I will investigate what type of functional responses and movement rules should be applied to fisheries. Indeed without considering individual level behavior over space and just analyzing the traditional Schaefer-Gordon fisheries model with constant effort it's obvious how changing the functional response may change the stability of the fishery system and give rise to the existence of alternative stable states (see previous section).
	
\subsection{Proposed Methods -- Theoretical}
Here I focus on encounter rates for fishermen, and how they vary with technology (distance at which fishermen can spot fish and handling time) ecology of the targeted species (clustering over space and movement), and what information a fishermen has regarding the abundance and distribution of the fish. Previous predator-prey metapopulation modeling has shown the importance of the choice of predator's movement rules and functional responses for system stability \citep{Bernsteinetal:1999}, along with the importance of the assumptions concerning the amount of information a predator has regarding the abundance and distribution of prey across the landscape \citep{Morozovetal:2012, Abramsetal:2011, Ruokolainenetal:2011, Anderson:2010, Abrams:2010, Matsumuraetal:2010}. Recent work has highlighted the importance of the mechanisms underlying encounter rates and combining both functional responses and movement rules and allowing them to emerge endogenously \citep{Anderson:2010, Gurarieetal:2013}. These models set up a framework that have the following sets of interactions and processes: 1) prey grow (typically logistically) within a patch and may or may not move between them; 2) predators move between patches, and search for prey within them; and 3) these predators may or may not interfere with one another. Given that the choice of functional response and movement rule typically determines stability in these models, I would like to examine how fishermen should be incentivized to search for fish, i.e. allocate effort over space with specific attention to how individually optimal behavior scales up to create population level movement rules and functional responses. 

My approach will be to use game-theoretic models to understand what type of searching behavior is most profitable. The problem fishermen face is for how long to search a given fishing ground (patch) before moving to the next one.  This is the giving-up-time (GUT) from the marginal value theorem \citep{Charnov:1976}. This theorem posits that a forager should leave the current patch when the marginal intake rate in the present patch is lower than the long-term system average across all patches. Understanding what the optimal giving-up-time is for fishermen, and how that depends on their gear, the ecology of the fish, and the spatial structure of the patches, and the information fishermen have on the distribution of fish across patches will provide information necessary to parameterize the metapopulation predator prey models. An additional need in these models is to understand how fishermen encounter fish, as this will largely determine the functional response. Fish are a depletable resource, thus how encounter rate should vary with depletion of patches will be an important consideration \citep{Gurarieetal:2013, MorozovPoggiale:2012, Anderson:2010, HutchinsonWaser:2007}. 

Finally with a functional form and movement rule in hand, I can parameterize the meta-population predator prey model and examine the system to understand under which conditions the system is stable. This will allow me to vary parameters relating to technology: fish-finder radius, speed; parameters relating to information: how well do fishermen know the conditions of other patches, and the ecology of the fish: intrinsic growth rate, dispersal patterns; in order to understand how the stability of the system depends on these system characteristics. 

\subsection{Proposed Methods -- Empirical}
These theoretical explorations of giving up time and encounter rate will be parameterized with the California groundfish fishery in mind. This is a multispecies fishery in which fishermen are licensed to fish a variety of groundfish species. The gear are predominantly trawlers and purse seiners, although hook and line and more conservative scottish-seiners are being incentivized to reenter the fishery. Specifically I hope to have catch data from The Nature Conservancy's Central Coast Groundfish Project. This program was motivated by the high rates of bycatch of vulnerable and endangered rockfish and habitat degradation from extensive trawling \citep{GroundfishTNC}. The management regime currently set up involves bycatch quotas assigned to individual fishermen. If a fishermen fishes over his bycatch quota, regardless of whether he has caught his target-species quota, he must stop fishing for the season \citep{CommercialDigest:2012}. 

In 2006 The Nature Conservancy bought 13 quota permits for the Half-Moon bay fishery, buying out a number of fishermen in order to reduce fishing pressure on vulnerable rockfish. TNC then leased these quotas back out to fishermen so long as they agreed to share data on where and when they caught bycatch. TNC has been using this data to create bycatch hotspot maps, and to try to understand where and when bycatch are most likely to be caught. This data provides vessel ID, location and date, along with composition of catch for an as yet to be determined number of fishermen. This data will allow me to directly estimate key behavioral parameters such as time to first encounter, average encounter rate, traveling time between encounters, and spatial autocorrelation in catch locations \citep{ecatch}. 

\subsection{Expected Results}
With this work I expect to be able to determine how individually incentivized searching and harvesting behavior for fishermen should scale into functional forms and movement rules. This will allow me to parameterize a metapopulation predator prey model and examine under what conditions the system is stable. Further, I can then examine how closing a patch to predation (fishing) pressure will change system dynamics, specifically considering whether the patch is a source or sink habitat, and how dispersal characteristics of the target fish affect the results. 
\section{Does the composition of the fleet affect how strategy should change in response to a MPA?}

\subsection{Introduction}
My previously proposed modeling framework does not take into account how fishermen's harvesting strategies may change in response to heterogeneous fleet composition. This requires consideration of how fishermen interact and change their strategy dynamically. I can then consider how strategy is dependent on the composition of the fleet. Previous work has found that fishermen may change gear not only to optimize catch rate, but to outcompete one another \citep{Rijnsdorpetal:2008}. Some empirical surveys of cooperatives exist \citep{Ovando2013132} and economic analyses have shown how cooperative strategies should evolve in marine common pool resources due to economic incentives \citep{KittsEdwards:2003}. Thus there's evidence that these systems do evolve and in response to the composition of the fleet. I am interested in understanding how harvesting strategy should change depending on the competitors that a fisherman has as fleet-mates. 

This approach will combine analytical modeling, agent-based simulation and empirical data analysis. Considering the groundfish parameterization discussed in the previous section, fishermen can be of many types: trawlers, scottish seiners, hook and line, among others. Boat sizes can vary putting different constraints on costs, information may be shared across groups allowing more or less accurate expectations of productivity, and the fishermen themselves may be operating individually or as part of a firm or cooperative. Given the possible combinations of gear types and information sharing, I would like to determine whether the optimal harvesting strategies should vary given the composition of the fleet of which a fisherman is part. Therefore the main question is: given that there's a discrete set of strategy types in the fishery, how do fishing strategies differ from one another, what are the optimal strategies to play in the fishery given the present composition of the fleet, how does the optimal strategy change as fleet composition changes, and whether fishermen do play the optimal strategy in the Half-Moon Bay fishery. My goal is to develop a game-theoretic model that allows fishermen to change strategy dynamically depending on conditions and guide testing in agent based model (ABM). I will analyze data from the Nature Conservancy (TNC) at Half Moon Bay using a random-utility-model (RUM) and cluster analysis to determine if the strategies played by individuals in the fishery match predictions from my modeling work.

\subsection{Methods -- Empirical}
Each fishing strategy can be characterized by a set of locations, gear types, times of year, and species targeted. In order to determine the existing strategies played in the Half-Moon Bay fleet, I will use a cluster analysis on the catch data similar to the analysis used by \cite{PelletierFerraris:2000} to define ``fishing tactics.'' The data should be in the form of catch location, species composition of catch, time and date. These catch data are also linked to vessel characteristics including size of boat and gear type. Using these characteristics as explanatory variables I will investigate how they relate to one another and whether fishing strategies cluster non-randomly. 

Here the target species is not known from the catch data, but if I assume that fishermen choose locations based on target species, I should be able to determine the target species from the species composition of the catch, specifically making the assumption that the species that forms the majority of catch is the species targeted unless explicitly noted otherwise in the catch logs. In many commercial logbook datasets, fishing location can only be determined down to a management statistical area, and fishermen are only required at minimum to record a single statistical area in which they fished. Thus if a fishermen fishes in multiple statistical areas, it is impossible to say with certainty whether the entire catch was indeed caught at the location recorded, and thus impossible to match species and biomass caught with location data. Luckily, as part of the TNC Central Coast Groundfish Project fishermen are required to log all catches. These catches are recorded in eCatch, a database program maintained by TNC. This ensures that if multiple catches are made in a single trip, all locations and species biomass for each catch are available for analysis. 

In order to provide a reduced set of descriptions for the catch data set and to determine the relationships between retained variables I will use factorial analyses (PCA, MCA) to give a geometric representation of the data. This will provide me the most important variables to characterize fishing trips. I will then analyze the coordinates of the data using classification techniques to group them into clusters which are the most clearly defined and homogenous. 

With fishing strategies characterized by location, target species, gear and time of year I can determine how fishermen should optimally allocate effort among these different strategies. For example \cite{MurawskiFinn:1986} considered how fishermen should optimally allocate effort among different fishing strategies in a mixed fishery and only considered the target species catchability and management restrictions on total allowable harvest. \cite{MarchalHorwood:1996} determined how fishermen should optimally allocate effort among a number of fishing tactics using a simulation model. 

\subsection{Expected Results}
The game theoretic model will provide predictions for how different harvesting strategies should trade off against one another. The agent based model will provide a venue in which to test how shifting different strategies plays out dynamically in explicit space. Comparing the game theoretic model with the agent based model can give insights to make sure the analytical approximation is appropriate. Using a RUM with the Half-Moon bay dataset will provide a test to see if the factors that are used in decision making can simulate what closing an area should do to the actual fishery. 
\subsection{Introduction}
Artisanal fisheries are important biologically, economically, socially and have been massively understudied \citep{Wormetal:2009}. Recent global syntheses pinpoint artisanal fisheries as typically overfished and degraded \citep{McCluskeyetal:2008}. There is a thorny problem of how to manage natural resources when poor infrastructure, poverty and weak governments prevail \citep{McCluskeyetal:2008}. There are some meta-analyses of how social and demographic characteristics co vary with the ecological response to MPAs in artisanal fisheries, but the results were not intuitive, population density -- expected to be negatively correlated with ecological response -- varied by region \citep{Pollnacetal:2010, Dawetal:2011}. Many higher-order social explanations have been proposed, i.e. \citep{Pollnacetal:2001, McClanahanetal:2006, Pollnacetal:2000} but given that harvesting strategies may make a difference in models considering commercial fisheries, do these same results apply to artisanal settings? 

Differences in artisanal fisheries as compared to commercial ones are great. Artisanal fisheries are defined as boats with non-industrial gear, target many different species typically for subsistence or local markets, and have a season goes year-round,  and feature limited processing or storage facilities (FAO 2006). Fuel costs make up significant portion of costs in artisanal fisheries (FAO 2006). Rapidly increasing sophistication and access to technology, even in developing countries, is changing the nature of fisheries (FAO 2006). Communications technology, including mobile phones allows unprecedented rapid dissemination about fishing sites and fish movements as well as marketing opportunities. Navigation technology including GPS, echo sounders and chart plotters are permeating developing country fisheries, potentially radically altering the information environment in which fishers make spatial decisions \citep{Dawetal:2010}. MPAs are also proposed as important management tools, perhaps even more so in artisanal, and typically tropical, fisheries. MPA recommendations mirror those in temperate, commercial fisheries where the suggestions are mostly based on biological characteristics of the target species, although there are a few examples of frameworks that consider fishermen preferences in making recommendations \citep{Tehetal:2012}. 

The question is, given these constraints what does the previous framework predict in terms of how fishing strategy should change and how that depends on others' strategies out on the water. There has been some previous work applying foraging models to artisanal fisheries, in particular using the ideal free distribution as a framework \citep{Abernethyetal:2007, Aswani:1998}. Other relevant work includes {Cinner 2009, who looked at how fish population on reefs varied with social metrics, and found that reef complexity was important. This suggests that there may be some de-facto reserves and that spatial heterogeneity in harvesting patterns may be at work. Few studies have been conducted in developing countries. Although there are a few studies that have considered how fishing strategy may behave in artisanal fisheries, these have been simulations or short-term random utility models \citep{LaloeSamba:1991, BeneTewfik:2001, Daw:2008}. 

\subsection{Methods}
The approach will be to adapt the theoretical and agent based model to replicate the major characteristics of an  artisanal fishery. Possible ideas include using the \citep{Pollnacetal:2010} study which found no clear pattern in MPA success to see if the fisheries considered vary broadly along dimensions that are likely to affect harvesting strategy. A case study will likely need to be chosen, but will compare between commercial and artisanal results, and ideally will be able to combine with fieldwork or catch data. I am still exploring that option. An ideal data collection method would put GPS on boats to see where they go, record catch and do RUM model similar to Half-Moon Bay data set. 

\subsection{Expected Results}
Comparisons between parameter dependence of results between commercial and artisanal fisheries and with meta-analysis results for added explanation. 

\chapter{Conclusion}Economic modeling such as the population level random-utility-models can only consider changes in harvesting behavior immediately after MPAs have been established, but can't necessarily predict longer term distributions of effort \citep{KahuiAlexander:2008}. Providing a mechanistic explanation of harvesting behavior allows an examination of fisheries over longer-time scales and may allow qualitative management suggestions. There have been some empirical examples of fishing changing the spatial distribution of fish \citep{Smithetal:2013} and catchability changing as a result of MPAs with fish inside reserves sometimes proving naive and thus easier to catch when they move out into fishable waters \citep{JanuchowskiHartleyetal:2012}. Other considerations include the fact that recovery trajectories after fishing may be different depending on community ecology \citep{Collieetal:2012}. Thus understanding how fishermen's incentives change over time is an important consideration in evaluating how these longer term ecological effects will intersect with harvesting practices and may change ecological and economic stability of fishery systems on longer time scales. 

\bibliographystyle{cbe}
\small
\bibliography{unread}

\end{document}
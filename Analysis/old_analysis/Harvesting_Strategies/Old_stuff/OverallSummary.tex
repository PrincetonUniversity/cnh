\documentclass[12pt, a4paper]{article}
\usepackage[margin=0.75in]{geometry} 	% see geometry.pdf on how to lay out the page. There's lots.
% \geometry{landscape} 	% rotated page geometry

%%% PACKAGES
\usepackage{booktabs} 	% for much better looking tables
\usepackage{array} 		% for better arrays (eg matrices) in maths
\usepackage{paralist} 	% very flexible & customisable lists (eg. enumerate/itemize, etc.)
\usepackage{verbatim} 	% adds environment for commenting out blocks of text & for better verbatim
\usepackage{subfigure} 	% make it possible to include more than one captioned figure/table in a single float
\usepackage{graphics}	% Show images with \includegraphics
\usepackage{url}		% Insert URL: \url{http:\\...}
\usepackage{natbib}		% Improved citations
\usepackage{ams math}	% Useful symbols, environments

%%% HEADERS & FOOTERS
\usepackage{fancyhdr} 	% This should be set AFTER setting up the page geometry
\pagestyle{fancy} 		% options: empty , plain , fancy
\renewcommand{\headrulewidth}{0pt} 	% customise the layout...
\lhead{}\chead{}\rhead{}
\lfoot{}\cfoot{\thepage}\rfoot{}

%%% SECTION TITLE APPEARANCE
\usepackage{sectsty}
%\allsectionsfont{\sffamily\mdseries\upshape} 	% (See the fntguide.pdf for font help)
% (This matches ConTeXt defaults)

%%% ToC APPEARANCE
\usepackage[nottoc,notlof,notlot]{tocbibind} 	% Put the bibliography in the ToC
%\usepackage[titles]{tocloft} 				% Alter the style of the Table of Contents
%\renewcommand{\cftsecfont}{\rmfamily\mdseries\upshape}
%\renewcommand{\cftsecpagefont}{\rmfamily\mdseries\upshape} % No bold!

%% END Article customise


\title{Background for Empirical Data}
\author{Emma Fuller}
\date{} 	% delete this line to display the current date

%%% BEGIN DOCUMENT
\begin{document}


\maketitle

\noindent The goal of this document is to be a repository for all my findings about the fisheries from which my datasets are generated.


\tableofcontents


\section{Introduction}
As part of my dissertation, I want to take quantify what strategies fishermen use to search for fish. I  have access to VMS  NOAA data from a number of west coast ground-fish fisheries. To be able to generate comparable theoretical data I need to know 

\begin{itemize}
\item the natural history of the fishery (ecology of targeted fish and larger ecosystem) 
\item the type of gear and history of regulations and species targeted, 
\item and the resolution of the data, what are the error rates? Accuracy?
\end{itemize}

\noindent By knowing the type of data I expect to get, I can build models that will generate similar data. I expect that these empirical data will be the test (and hopefully validation) of my theoretical predictions. 


\section{NOAA Data}

\subsection{Introduction}
% What fisheries? What resolution?


\bibliographystyle{cbe}
\bibliography{}
\end{document}
\documentclass[12pt, a4paper]{article}
\usepackage[margin=1in]{geometry} 	% see geometry.pdf on how to lay out the page. There's lots.
% \geometry{landscape} 	% rotated page geometry

%%% PACKAGES
\usepackage{booktabs} 	% for much better looking tables
\usepackage{array} 		% for better arrays (eg matrices) in maths
\usepackage{paralist} 	% very flexible & customisable lists (eg. enumerate/itemize, etc.)
\usepackage{verbatim} 	% adds environment for commenting out blocks of text & for better verbatim
\usepackage{subfigure} 	% make it possible to include more than one captioned figure/table in a single float
\usepackage{graphics}	% Show images with \includegraphics
\usepackage{url}		% Insert URL: \url{http:\\...}
\usepackage{natbib}		% Improved citations
\usepackage{amsmath}	% Useful symbols, environments

%%% HEADERS & FOOTERS
\usepackage{fancyhdr} 	% This should be set AFTER setting up the page geometry
\pagestyle{fancy} 		% options: empty , plain , fancy
\renewcommand{\headrulewidth}{0pt} 	% customise the layout...
\lhead{}\chead{}\rhead{}
\lfoot{}\cfoot{\thepage}\rfoot{}

%%% SECTION TITLE APPEARANCE
\usepackage{sectsty}
%\allsectionsfont{\sffamily\mdseries\upshape} 	% (See the fntguide.pdf for font help)
% (This matches ConTeXt defaults)

%%% ToC APPEARANCE
\usepackage[nottoc,notlof,notlot]{tocbibind} 	% Put the bibliography in the ToC
%\usepackage[titles]{tocloft} 				% Alter the style of the Table of Contents
%\renewcommand{\cftsecfont}{\rmfamily\mdseries\upshape}
%\renewcommand{\cftsecpagefont}{\rmfamily\mdseries\upshape} % No bold!

%% END Article customise


\title{Human harvesting behavior in MPA studies}
\author{Emma Fuller}
\date{} 	% delete this line to display the current date

%%% BEGIN DOCUMENT
\begin{document}


\maketitle

The goal of this paper is to collect all notes related to understanding how human behavior is incorporated in MPA modeling and theory. 

\tableofcontents


\section{Introduction}
The argument of my generals proposal was that MPAs are suggested using overly simplistic assumptions about how fishermen will reallocate effort after an MPA has been implemented. Better understanding how fishermen should allocate their effort after an MPA may provide different results than MPAs that have been previously suggested. The goal of this document is to figure out the answers to the following

\begin{itemize}
\item How do MPA models take human harvesting into account?
\item Specifically how do MPAs predict that human harvesting will change after an MPA is put in? 
\item How do empirical studies control for changes in spatial patterns of human harvesting
\item What are the generally known facts about MPAs and their establishment? What's the state of knowledge?'
\item Consider Heather Leslie and Tim Daw's work, how have they integrated artisanal fishing and MPAs?
\end{itemize}

\noindent Understanding more completely how human behavior is incorporated into MPA studies will help me better understand what assumptions are made and how I can test them. 

\bibliographystyle{cbe}
\bibliography{}
\end{document}
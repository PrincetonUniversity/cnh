\documentclass[]{article}
\usepackage[T1]{fontenc}
\usepackage{lmodern}
\usepackage{amssymb,amsmath}
\usepackage{ifxetex,ifluatex}
\usepackage{fixltx2e} % provides \textsubscript
% use upquote if available, for straight quotes in verbatim environments
\IfFileExists{upquote.sty}{\usepackage{upquote}}{}
\ifnum 0\ifxetex 1\fi\ifluatex 1\fi=0 % if pdftex
  \usepackage[utf8]{inputenc}
\else % if luatex or xelatex
  \usepackage{fontspec}
  \ifxetex
    \usepackage{xltxtra,xunicode}
  \fi
  \defaultfontfeatures{Mapping=tex-text,Scale=MatchLowercase}
  \newcommand{\euro}{€}
\fi
% use microtype if available
\IfFileExists{microtype.sty}{\usepackage{microtype}}{}
\usepackage{natbib}
\bibliographystyle{plainnat}
\ifxetex
  \usepackage[setpagesize=false, % page size defined by xetex
              unicode=false, % unicode breaks when used with xetex
              xetex]{hyperref}
\else
  \usepackage[unicode=true]{hyperref}
\fi
\hypersetup{breaklinks=true,
            bookmarks=true,
            pdfauthor={},
            pdftitle={},
            colorlinks=true,
            urlcolor=blue,
            linkcolor=magenta,
            pdfborder={0 0 0}}
\urlstyle{same}  % don't use monospace font for urls
\setlength{\parindent}{0pt}
\setlength{\parskip}{6pt plus 2pt minus 1pt}
\setlength{\emergencystretch}{3em}  % prevent overfull lines
\setcounter{secnumdepth}{0}

\author{}
\date{}

\begin{document}

\section{Annotated Bibliography}

\subsection{MPAs}

What is most current about MPAs?

\subsubsection{Do MPAs work?}

Are they context dependent in any way (depends on the species targeted,
type of habitat (coral reef or temperate), what the goals are)? How is
`success' measured? Is there differences in success rate between MPAs
that focus on conservation goals versus fishery goals? What are the
common goals? How do people choose where MPAs go? How to fishermen
respond to MPAs being set up and how are they included in the planning
process? Ideal to know if fishing outside the reserve is taken into
account.

\paragraph{References}

\begin{itemize}
\itemsep1pt\parskip0pt\parsep0pt
\item
  \href{http://marineprotectedareas.noaa.gov/pdf/helpful-resources/do_no_take_reserves_benefit_adjacent_fisheries.pdf}{NOAA
  fact sheet about MPAs}: Has a bit about when MPAs will be beneficial
  for fishermen. References about movement, looks like recruitment is
  all that's considered. Citations inside.
\item
  Hilborn (when will reserves be useful?)
\item
  Gaines
\item
  Botsford
\item
  Wilen
\item
  Baskett
\item
  Halpern (empirical synthesis)
\end{itemize}

\subsubsection{Spatial predator-prey work: theoretical}

How is space predicted to control stability of predator-prey models?
What are the conditions for stability or instability in spatial-predator
prey models. And how is that different from what's expected if there's
no spatial structure? Are their connections between predictions for
stability in theoretical models and fishermen scenarios?

\paragraph{References}

\begin{itemize}
\itemsep1pt\parskip0pt\parsep0pt
\item
  Encyclopedia of Ecology?
\item
  Kareiva/Tilman work?
\end{itemize}

\subsubsection{How do fishermen distribute their effort in space}

Do fishermen array themselves non-randomly in space. Are there general
reviews of how fishermen array themselves in space or just case studies?
Are there generalities to be pulled out?

\begin{itemize}
\itemsep1pt\parskip0pt\parsep0pt
\item
  Rum models
\item
  {[}Theories and behavioral drivers underlying fleet dynamics
  models{]}(sente:/Emma's+reading/vanPuttenetal:2011): Generalities for
  spatial distribution of effort \citet{vanPuttenetal2011}.
\end{itemize}

\bibliography{refs}

\end{document}

\documentclass[a4paper,10pt]{article}
\let\oldmarginpar\marginpar
\renewcommand\marginpar[1]{\-\oldmarginpar[\raggedleft\footnotesize #1]%
{\raggedright\footnotesize #1}}
\usepackage{charter}
\usepackage{graphicx}
\usepackage{amsmath}
\usepackage{amssymb}
\usepackage{mathtools}
\usepackage{array}
\usepackage{natbib}
\usepackage{gensymb}
\usepackage{epstopdf}
\usepackage{multirow} %for multi column spanning rows or multi row spanning columns
 \usepackage{rotating} %for rotating tables
 \usepackage{subfig} %to have figures within figures
\usepackage{multicol}
 \usepackage{ulem} % Allows for underlining without weird formatting glitches. 
 \usepackage{pifont}% Allows for use of ZipfDingbats font
\usepackage[usenames,dvipsnames,svgnames,table]{xcolor} % Allows colored font
\usepackage{hyperref} %allows hyperlinks in the document
\usepackage[linecolor=black,backgroundcolor=white,bordercolor=white,textsize=small]{todonotes}
\usepackage{wrapfig}
\usepackage[top=0.5in, bottom=0.75in, outer=1.5in, inner=0.5in, heightrounded, marginparwidth=1in, marginparsep=0.25in]{geometry}
\usepackage{marginnote}
 \begin{document}

\subsubsection*{Fishing in Alaska, and the sharing of information: \cite{orth1987fishing}}
\begin{itemize}
\item Response to {\color{Grey}Gatewood 1984}, factors affecting spatial location of harvest in purse-seine fishery in Southeast Alaska include:
	\begin{itemize}
	\item Preference for particular area (experience?)
	\item Response to fluctuations of salmon run
	\item Distribution of seine fleet
	\item Management decisions
	\item Weather
	\end{itemize}

\item In author's experience, the general differences that exist within the fishery between ares in Southeast Alaska are the type and relative importance of information used in deciding when and where to make a fishing set (the importance of seeing salmon jumping varies by location), greater exposure to weather and sea conditions (vessels fishing in rougher waters are larger and more technologically sophisticated) and the physiological condition of the salmon (as salmon move closer to natal stream the deteriorate physically and are economically less desirable), and the historical development of the purse seine fishery in different locations.  

\item {\color{Grey}Gatewood 1984} describes Information-Sharing Cliques (ISCs) as a ``small group of skippers who negotiate a limited form of cooperation in advance of some seine periods to share scouting reports on the number of fish and other boats seen in various areas,'' but Orth finds this interpretation as overly narrow and only representing one aspect of the ISCs. Instead he says that the ISCs primary function is the exchange of information during fishing periods, not before. Information concerning resource abundance and vessel density is continually communicated over the radio with ISCs while fishing. Identities, location of transmitter and receiver, and information are kept secret through the use of illegal radio frequencies, radio scramblers, and codes. 

\item Makes argument that ecological characteristics that may lead to formation of groups and influence their size are random or patchy distribution, variable abundance, high mobility, and low predictability. 

\item Skippers use a variety of factors to make patch choice
\begin{itemize}
\item tidal set and drift
\item presence of tidal streaks
\item salmon jumping
\item information from partners
\item observed catches
\item spatial distribution of other seiners.
\end{itemize}
In addition, the skipper constantly uses his own catch (success of previous set) to refine assessment of prey behavior and local abundance. Found that of these factors, information obtained from partners is important and was used 35\% of the time in deciding when and where to make a fishing set; n = 172 fishing sets (doesn't say how many boats though). 
\item Scouting trips and rendezvous prior to a fishing period is probably a direct response to the time interval between seine openings and pervious fishing period. The functional relationship between relative value and age of information most likely operates in a stepwise manner. The threshold for when a scouting trip is initiated by a skipper likes varies among skippers, depending on additional factors, such as skill of the skipper, technological capabilities, and the harvest to date relative to an individual's expectations. So scouting trips are only initiated when time between fishing is long. 
\item {\color{Grey}Gatewood 1984} says not all skippers participate in ISCs and if they do, they only participate in one. Orth says no, all participate in ISCs and may participate in more than one. Additionally, novice seiners have crewed for other boats for 4-10 years before having a boat of their own, and so have a novice-mentor relationship and are included in his ISC.
\item {\color{Grey}Gatewood 1984} hypothesizes that the size of an ISC is determined by 1) the job of that the ISC is trying to accomplish, 2) the number of individuals needed to effectively scout an area, 3) the hook off. Orth instead argues that while partners in ISCs are often fishing close to one another (approx 90\% of the time), that they rarely share the same set. Instead he argues that the maximal size of an ISC is determined by the effective control of information. Every skipper, he says, has obligations and commitments outside of his ISC. Information has a way of leaking out to these other individuals. As the number of individuals with access to the same information increases, the value of the information decreases. If information in consistently inaccurate or a partner is too liberal with sharing it, the individual may be eliminated from the ISC for his transgressions. 
\item Also the occurrence of information-sharing groups (code groups or information networks) in commercial fisheries is not an uncommon form of social organization. These groups have been documented in the Newfoundland deep-sea trawl fishery, Kattegat trawl fishery of Sweden, the albacore troll fishery and swordfish fishery of California and the Eastern Pacific  tuna seine fishery. 
\item The development of information networks is a rational and predictable response to a situation where environmental risk and uncertainty are the norm. In the salmon seine fishery of Southeast Alaska, the purpose of information networks is to increase search efficiency by communicating information concerning resource abundance and distribution, and the distribution of other seiners. These groups function primarily while fishing, through when conditions warrant, scouting trips are conducted prior to a fishing period. 
\end{itemize} 

\bibliographystyle{cbe}
\bibliography{refs}

\end{document}

\bibliographystyle{cbe}
\bibliography{refs}

\end{document}

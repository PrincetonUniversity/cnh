\documentclass[a4paper,10pt]{article}
\let\oldmarginpar\marginpar
\renewcommand\marginpar[1]{\-\oldmarginpar[\raggedleft\footnotesize #1]%
{\raggedright\footnotesize #1}}
\usepackage{charter}
\usepackage{graphicx}
\usepackage{amsmath}
\usepackage{amssymb}
\usepackage{mathtools}
\usepackage{array}
\usepackage{natbib}
\usepackage{gensymb}
\usepackage{epstopdf}
\usepackage{multirow} %for multi column spanning rows or multi row spanning columns
 \usepackage{rotating} %for rotating tables
 \usepackage{subfig} %to have figures within figures
\usepackage{multicol}
 \usepackage{ulem} % Allows for underlining without weird formatting glitches. 
 \usepackage{pifont}% Allows for use of ZipfDingbats font
\usepackage[usenames,dvipsnames,svgnames,table]{xcolor} % Allows colored font
\usepackage{hyperref} %allows hyperlinks in the document
\usepackage[linecolor=black,backgroundcolor=white,bordercolor=white,textsize=small]{todonotes}
\usepackage{wrapfig}
\usepackage[top=0.5in, bottom=0.75in, outer=1.5in, inner=0.5in, heightrounded, marginparwidth=1in, marginparsep=0.25in]{geometry}
\usepackage{marginnote}
 \begin{document}

\subsubsection*{The social aspects of fishing effort: \cite{Gezelius:2007}}
\cite{Gezelius:2007} argues that increased communication does not automatically result in improved governance of common pool resources. 
\begin{itemize}
\item The Fishery
	\begin{itemize}
	\item Atlantic blue whiting fishery
	\item The fishery did not have efficient quota regulations
	\item Subject to rapid growth in fishing effort making it the largest fishery in the Atlantic
	\item Norway's blue whiting has mainly been a directed pelagic trawl fishery performed with combined purse seiners/trawlers. 
	\item The season starts in the spawning areas in January/Febuary and ends in the Norwegian Sea in midsummer. 
	\item Vessels participate in several other fisheries during this time, such as purse seine fishing for mackerel and herring. 
	\item Industrial trawlers, a separate fleet segment, perform a mixed fishery  for blue whiting and Norway pout. 
	\item Profits depend on good catch rates and great volumes.
	\item Due to international policy on catch rates, the Norway ministry of fishing did not enforce its quotas. In 2003 the Norwegian fishery took almost twice their fishery before the Ministry of Fisheries stopped them. In 2004 the ministry granted Norwegian vessels a free fishery for blue whiting in Norwegian waters and on the high seas, so fieldwork for this study was carried out at a time when the fishery represented a case of rapidly increasing fishing effort int he absence of quota regulations. 
	\item The Norwegian authorities regulate entrance to the fishery via licensing. So the increased catches mainly come from increased fishing effort by a stable group of participants than from a large number of new entrants. 
	\end{itemize}
\item The fishermen
\begin{enumerate}
	\item More than 20 offshore purse seiners/trawlers
	\item 4,500 inhabitants in Seaborn Hills (location of fieldwork)
	\item 300 registered fishermen, and almost half the workforce is employed in fishing-related activities.
	\item Boats are mostly owned by local families and operated by crews of 8-10 people, including the skipper, net/trawl boss, mate/bosun, chief with two assistants, steward and a couple of fishermen. All employed on a share basis.
	\item Professional reputations of owners, skippers and trawl bosses greatly affect their standing in the community.
\end{enumerate}
\item Norms
\begin{itemize}
\item Social norms of excellence: underlie prestige of company, skipper and trawl boss. Related to success in fishing compared to others. Relative performance compared almost instantaneously (no waiting until you get back to shore)
\end{itemize}
\end{itemize}

\bibliographystyle{cbe}
\bibliography{refs}

\end{document}

\documentclass[a4paper,10pt]{article}
\let\oldmarginpar\marginpar
\renewcommand\marginpar[1]{\-\oldmarginpar[\raggedleft\footnotesize #1]%
{\raggedright\footnotesize #1}}
\usepackage{charter}
\usepackage{graphicx}
\usepackage{amsmath}
\usepackage{amssymb}
\usepackage{mathtools}
\usepackage{array}
\usepackage[bitstream-charter]{mathdesign}	% using charter font for math
\usepackage{natbib}
\usepackage{gensymb}
\usepackage{epstopdf}
\usepackage{multirow} %for multi column spanning rows or multi row spanning columns
 \usepackage{rotating} %for rotating tables
 \usepackage{subfig} %to have figures within figures
\usepackage{multicol}
 \usepackage{ulem} % Allows for underlining without weird formatting glitches. 
 \usepackage{pifont}% Allows for use of ZipfDingbats font
\usepackage[usenames,dvipsnames,svgnames,table]{xcolor} % Allows colored font
\usepackage{hyperref} %allows hyperlinks in the document
\usepackage[linecolor=black,backgroundcolor=white,bordercolor=white,textsize=small]{todonotes}
\usepackage{wrapfig}
\usepackage[top=0.5in, bottom=0.75in, outer=1.5in, inner=0.5in, heightrounded, marginparwidth=1in, marginparsep=0.25in]{geometry}
\usepackage{marginnote}
\usepackage{hyperref}
 \begin{document}

\section*{\href{run:Signalling and the evolution of cooperative foragi.pdf alias}{\color{Blue}Signaling and the evolution of cooperative foraging - Torney et al. 2011}}

\subsection*{Set Up}
Agents are following a resource gradient, can either signal when found resources, or not signal. If an individual finds a resource and is a signaler, they will signal. Individuals within signal radius will move towards signaler unless located at a resource. If an individual is not signaled to and not on a resource, they perform a correlated random walk through the environment. 

\subsection*{General Points}
\begin{itemize}
\item Main question is to see if for some regions of parameter space, find that a cooperative signaling strategy is stable. Show this numerically, then reduce the simulation to something analytically tractable and examine possible mechanistic reasons for such a strategy to be maintained.   
\item Social foraging theory might be useful: refs 12-15
\end{itemize}


\bibliographystyle{cbe}
\bibliography{refs}

\end{document}


\documentclass[]{article}
\usepackage[T1]{fontenc}
\usepackage{lmodern}
\usepackage{amssymb,amsmath}
\usepackage{ifxetex,ifluatex}
\usepackage{fixltx2e} % provides \textsubscript
% use upquote if available, for straight quotes in verbatim environments
\IfFileExists{upquote.sty}{\usepackage{upquote}}{}
\ifnum 0\ifxetex 1\fi\ifluatex 1\fi=0 % if pdftex
  \usepackage[utf8]{inputenc}
\else % if luatex or xelatex
  \ifxetex
    \usepackage{mathspec}
    \usepackage{xltxtra,xunicode}
  \else
    \usepackage{fontspec}
  \fi
  \defaultfontfeatures{Mapping=tex-text,Scale=MatchLowercase}
  \newcommand{\euro}{€}
\fi
% use microtype if available
\IfFileExists{microtype.sty}{\usepackage{microtype}}{}
\usepackage[margin=1in]{geometry}
\usepackage{longtable,booktabs}
\ifxetex
  \usepackage[setpagesize=false, % page size defined by xetex
              unicode=false, % unicode breaks when used with xetex
              xetex]{hyperref}
\else
  \usepackage[unicode=true]{hyperref}
\fi
\hypersetup{breaklinks=true,
            bookmarks=true,
            pdfauthor={Emma Fuller},
            pdftitle={Mgmt\_grp clusters},
            colorlinks=true,
            citecolor=blue,
            urlcolor=blue,
            linkcolor=magenta,
            pdfborder={0 0 0}}
\urlstyle{same}  % don't use monospace font for urls
\setlength{\parindent}{0pt}
\setlength{\parskip}{6pt plus 2pt minus 1pt}
\setlength{\emergencystretch}{3em}  % prevent overfull lines
\setcounter{secnumdepth}{0}

\title{Mgmt\_grp clusters}
\author{Emma Fuller}
\date{July 10, 2014}

\begin{document}

\begin{center}
\huge Mgmt\_grp clusters \\[0.2cm]
\large \emph{Emma Fuller}\\[0.1cm]
\large \emph{July 10, 2014} \\
\normalsize
\end{center}


\subsection{Methods}\label{methods}

Using all trips landed between 2009-2013 in Washington, Oregon and
California found proportion (in lbs) of landing that constituted of each
of 8 management groups assigned by PacFin.

\begin{longtable}[c]{@{}cc@{}}
\toprule\addlinespace
SPID & common\_name
\\\addlinespace
\midrule\endhead
CPEL & all coastal pelagic
\\\addlinespace
CRAB & all crab
\\\addlinespace
GRND & all groundfish
\\\addlinespace
HMSP & all highly migratory
\\\addlinespace
OTHR & other species (no m-group)
\\\addlinespace
SAMN & all salmon
\\\addlinespace
SHLL & all shellfish
\\\addlinespace
SRMP & all shrimp \& prawns
\\\addlinespace
\bottomrule
\addlinespace
\caption{Management groups}
\end{longtable}

Using this dataset, I first performed a PCA to reduce down to 5
principal components (retaining \textgreater{} 80\% of the original
variation) and then used \texttt{clara()}\footnote{will flesh this out}
to search for characteristic landings profiles. \texttt{clara()}
requires the number of existing clusters to given a priori, so I tried
from 1-30 possible clusters. Because clustering algorithms are
exploratory approaches, there is no ``correct'' number of clusters
known. In order to evaluate the fit of each possible cluster solution
(1-30) I look at the objective function and the average silhouette
width. The objective function measures how tightly clumped the data are
around each cluster center. The best clustering solutions are ones which
minimize this objective function. The average silhouette is a measure of
how well seperated each cluster is, and ranges from 0 (unseperated) to 1
(completely seperated). Clustering solutions with an average silhouette
width \textless{} 0.6 should not be considered.

\subsection{Results}\label{results}

Examining how the objective function and average silhouette width vary
with the number of clusters given to \texttt{clara()}, 8 clusters looks
like the minimum number of clusters which satisfies both minimizing the
objective function and maximizing the average silhouette width.

Finding that 8 clusters is the best fit is not surprising, as there are
8 different management groups.

The 8 cluster solution is futher coroborated by the fact that as we go
above 8 clusters, only a very small number of trips are placed in the
new clusters.

\subsubsection{Examining cluster
contents}\label{examining-cluster-contents}

\end{document}

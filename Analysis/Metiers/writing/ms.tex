\documentclass[]{article}
\usepackage{lmodern}
\usepackage{amssymb,amsmath}
\usepackage{ifxetex,ifluatex}
\usepackage{fixltx2e} % provides \textsubscript
\ifnum 0\ifxetex 1\fi\ifluatex 1\fi=0 % if pdftex
  \usepackage[T1]{fontenc}
  \usepackage[utf8]{inputenc}
\else % if luatex or xelatex
  \ifxetex
    \usepackage{mathspec}
    \usepackage{xltxtra,xunicode}
  \else
    \usepackage{fontspec}
  \fi
  \defaultfontfeatures{Mapping=tex-text,Scale=MatchLowercase}
  \newcommand{\euro}{€}
\fi
% use upquote if available, for straight quotes in verbatim environments
\IfFileExists{upquote.sty}{\usepackage{upquote}}{}
% use microtype if available
\IfFileExists{microtype.sty}{%
\usepackage{microtype}
\UseMicrotypeSet[protrusion]{basicmath} % disable protrusion for tt fonts
}{}
\usepackage[margin=1in]{geometry}
\usepackage{graphicx}
\makeatletter
\def\maxwidth{\ifdim\Gin@nat@width>\linewidth\linewidth\else\Gin@nat@width\fi}
\def\maxheight{\ifdim\Gin@nat@height>\textheight\textheight\else\Gin@nat@height\fi}
\makeatother
% Scale images if necessary, so that they will not overflow the page
% margins by default, and it is still possible to overwrite the defaults
% using explicit options in \includegraphics[width, height, ...]{}
\setkeys{Gin}{width=\maxwidth,height=\maxheight,keepaspectratio}
\ifxetex
  \usepackage[setpagesize=false, % page size defined by xetex
              unicode=false, % unicode breaks when used with xetex
              xetex]{hyperref}
\else
  \usepackage[unicode=true]{hyperref}
\fi
\hypersetup{breaklinks=true,
            bookmarks=true,
            pdfauthor={Emma Fuller},
            pdftitle={People in EBM},
            colorlinks=true,
            citecolor=blue,
            urlcolor=blue,
            linkcolor=magenta,
            pdfborder={0 0 0}}
\urlstyle{same}  % don't use monospace font for urls
\setlength{\parindent}{0pt}
\setlength{\parskip}{6pt plus 2pt minus 1pt}
\setlength{\emergencystretch}{3em}  % prevent overfull lines
\setcounter{secnumdepth}{0}

%%% Change title format to be more compact
\usepackage{titling}
\setlength{\droptitle}{-2em}
  \title{People in EBM}
  \pretitle{\vspace{\droptitle}\centering\huge}
  \posttitle{\par}
  \author{Emma Fuller}
  \preauthor{\centering\large\emph}
  \postauthor{\par}
  \predate{\centering\large\emph}
  \postdate{\par}
  \date{January 3, 2015}




\begin{document}

\maketitle


\section{Introduction}\label{introduction}

We perform a metier analysis on the US west coast commercial fisheries.
We demonstrate how this analysis improves upon previously used
definitions of fisheries in the US, and show the utility of these
definitions by building two different types of networks: a typical
foodweb, bipartite network between fisheries and species fished; and a
social-ecological network in which human linkages across fisheries are
included. We then apply network measures to demonstrate how
management-revelent social and ecological conclusions can be drawn from
this analysis. In particular, we highlight how this approach can be
useful to better include humans in ecosystem-based management.

\section{Methods}\label{methods}

We do a metier analysis using infoMap {[}more detail to come{]} and
examine the resulting fisheries-deliniations.

We build a bipartite network using the metiers where the interaction
strength between metier and species is the volume of species caught
averaged over the 5 year dataset.

We build a social-ecological network by adding a participation network
to the bipartite foodweb. The participation network has as its nodes,
metiers. The strength of connection between the fisheries is the number
of individuals which participated in both fisheries averaged over 5
years.

We first analyze the bipartite network to try to determine the following
types of inferences

\begin{itemize}
\itemsep1pt\parskip0pt\parsep0pt
\item
  vulnerability: highlight which species are more or less vulnerable
  (i.e.~targeted by more than one fishery)
\item
  generality: highlight which fisheries catch more or less species
\end{itemize}

And then analyze the social-ecological network and show how many of the
same metrics can be applied to an SES network.

\begin{itemize}
\itemsep1pt\parskip0pt\parsep0pt
\item
  centrality: which fisheries are important for everyone (i.e.~refuge
  fisheries)
\end{itemize}

\section{Results}\label{results}

\subsection{This analysis improves upon heuristic definitions of
fisheries}\label{this-analysis-improves-upon-heuristic-definitions-of-fisheries}

Commonly suggested heuristics to define species targeted often center on
species that are the majority of a trip's catch either by weight or
revenue. We compare our definitions of fisheries and find it reduces the
number of fisheries by about 80\% (from \textasciitilde{}500 to
\textasciitilde{}100).

\includegraphics{ms_files/figure-latex/unnamed-chunk-2-1.pdf}

Here I plot a matrix where columns are the metiers, and rows are
fisheries defined as major species by revenue. The cell shows the $\log$
number of trips in which that species was the majority catch by
revenue.\footnote{Using the Adjusted Rand Index ($ARI$) I find no
  difference between classification using volume or revenue ($ARI=$
  0.9909052)} Darker colors indicate many trips with this species as
majority, lighter colors indicate relatively few species as a majority.
The metiers are ordered using a detrended correspondance analysis
(\texttt{decorana} function in the \texttt{vegan} package to aid in
visualization, \texttt{R}). The plot demonstrates the reduction in
dimensionality as a number of rare species are grouped into a single
fishery.

\subsection{This analysis allows building of food webs that include
humans}\label{this-analysis-allows-building-of-food-webs-that-include-humans}

Ecosystem based management strives to manage an ecosystem holistically,
and include humans in the analysis (Larkin 1996; Lucey and Fogarty
2013). Here we demonstrate such a foodweb.

\includegraphics{ms_files/figure-latex/unnamed-chunk-3-1.pdf} Here I
plot a bipartite network in which upper nodes are metier-defined
fisheries and the lower nodes are fish species. The strength of
interaction is to the total volume harvested over the 5 year dataset.
For visual clarity, this network is retricted to showing the top 10
fisheries by volume and the species harvested. Further the species
caught are restricted to those which had at least an average of 20 lbs
caught per year.

Using this network, we can estimate the vulnerability of species by the
number of seperate fisheries which exploit them, and the intensity with
which they do so.

\begin{verbatim}
##    generality.HL vulnerability.LL 
##         1.540513         1.032836
\end{verbatim}

\subsection{Can improve this web by considering human-mediated
connectivity between
fisheries}\label{can-improve-this-web-by-considering-human-mediated-connectivity-between-fisheries}

\section*{References}\label{references}
\addcontentsline{toc}{section}{References}

Larkin, Peter Anthony. 1996. ``Concepts and Issues in Marine Ecosystem
Management.'' \emph{Reviews in Fish Biology and Fisheries} 6 (2).
Springer: 139--64.

Lucey, Sean M, and Michael J Fogarty. 2013. ``Operational Fisheries in
New England: Linking Current Fishing Patterns to Proposed Ecological
Production Units.'' \emph{Fisheries Research} 141. Elsevier: 3--12.

\end{document}
